migliorare la parte di ARSAPD nello stato dell'arte perché non parliamo proprio dell'algoritmo che usa!!! !! ! ! ! ! 
sistemare la parte sulla analisi acustica dei suoni respiratori
si può usare un algoritmo di onset detection basato sulla differenza rispetto ad un certo modello di previsione?
aggiungere una parte nella quale si cerca di distinguere la inspirazione dall'espirazione
aggiungere una parte nella quale si cerca di riconoscere dei pattern respiratori tipici di una condizione pre infarto se esistono
aggiungere riconoscimento di Cheyne–Stokes e aggiungere a che serve
aggiungere citazioni ovunque
se mi rimane molto tempo: cercare di adattare Phoneme Recognition Using Time-Delay Neural Network per il riconoscimento del respiro
cambiare il colore delle figure in blu leggero e/o verde leggero
aggiungere lo stato dell'arte sulla classificazione dei suoni respiratori
controllare lo spelling di Hamming Hanning
riassumere la parte A 1) riguardo le envelope dall'articolo: a tutorial on onset detection in music signal.
sistemare la bibliografia: rimuovere gli elementi non usati, sistemare la scrittura
cos'è la Parenchima polmonare ?  ? ? ? ? ?
correggere l'ortografia degli accenti 
http://it.wikipedia.org/wiki/Ortografia_italiana#L.27accento_grafico
indicare per tutte le immagini dove le abbiamo prese


riassumere meglio ACOUSTICAL RESPIRATORY SIGNAL ANALYSIS AND PHASE DETECTION
riassumere meglio ACOUSTICAL RESPIRATORY SIGNAL ANALYSIS AND PHASE DETECTION
perche' manca l'algoritmo
riassumere meglio ACOUSTICAL RESPIRATORY SIGNAL ANALYSIS AND PHASE DETECTION
riassumere meglio ACOUSTICAL RESPIRATORY SIGNAL ANALYSIS AND PHASE DETECTION



controllare l'uniformità dei tempi verbali
finire l'interfaccia grafica
parlare di rumore bianco/ additivo/ lineare ...
parlare di apnea inspiratoria e di apnea espiratoria...
discutere i possibili protocolli di rete che si possono usare per prendere i dati da uno stetoscopio o da una banca dati remota
http://en.wikipedia.org/wiki/Streaming_media#Protocol_problems
migliorare la parte sulla valutazione dell'output: falsi negativi ecc...
l'interruzione del sonno con una allarme e' un beneficio o un maleficio?
fare una analisi della complessità computazionale dell'algoritmo rispetto alla macchina virtuale java.
eliminare se possibile tutti gli usi di classi cosi' da poter permettere una compilazione diretta e bypassare la JVM.
quanto impiega la cpu in percentuale? 
si può usare un meccanismo di feedback per modificare i parametri dei filtri o dei clustering?
spiegare meglio le finestre
pacchetti wavelet
aggiungere un esempio di clustering con una chiara immagine che mappa una sequenza di beat in una sequenza di cluster
aggiungere la complessità computazionale dell'algoritmo
aggiungere un capitolo nel quale si riassumono le tecniche di separazione dei suoni del cuore dai suoni dei polmoni
finire il capitolo di test, manca la complessità dell'algoritmo e l'errore nei casi di test dei file
aggiungere una descrizione di rumore bianco, rosa e marrone
finire la parte sulla soglia 
aggiungere la parte sulla scelta del protocollo di acquisizione dati dallo stetoscopio