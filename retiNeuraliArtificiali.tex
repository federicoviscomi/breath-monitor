\section{Reti neurali}
\label{retineurali}
Le \emph{reti neurali artificiali} sono modelli matematici e computazionali della corteccia cerebrale. 
Gli elementi di base di una rete neurale artificiale sono:
\begin{description}
  \item[neuroni artificiali o unita']
%     neuroni artificiali o \emph{unita'} 
  \item[collegamenti]
    I neuroni artificiali sono connessi attraverso \emph{collegamenti}. 
    Ogni collegamento ha un \emph{peso} numerico, in particolare il collegamento tra l'unita' $i$ e l'unita' $j$ ha peso $w_{i,j}$. 
\end{description}
Una certa unita' $i$ pu\`o inviare un \emph{segnale} o \emph{attivazione} $a_{i}$ su tutti i suoi collegamenti. 
Il peso di un collegamento determina la forza e il segno del segnale. 
Assumiamo l'esistenza di una certa unita' di input che manda il segnale $a_{0}$. 
Ogni unita' si comporta nel modo seguente:
\begin{enumerate}
  \item 
    calcola la somma pesata dei suoi input
    \begin{center}
      $in_{j} = \sum\limits_{i=0}^n w_{i,j} a_{i}$
    \end{center}
  \item
    calcola il proprio output in base al valore calcolato al punto precedente e ad una \emph{funzione di attivazione} $g$: 
    \begin{center}    
      $a_{j}=g(in_{j})$
    \end{center}    
\end{enumerate}
La funzione di attivazione di solito \`e una funzione del tipo:
\begin{center}
  \begin{tabular}{lll}
      $g(x)= \left\{ 
	  \begin{array}{ll}
	      0
	    &
	      if\; x< 0
	    \\
	      1
	    &
	      if\; x> 0
	  \end{array}
      \right.$
    &
      o
    &
      $g(x)= \frac{1}{1+e^{-x}}$
  \end{tabular}
\end{center}

Ci sono due tipi di reti neurali:
\begin{description}
  \item[feed forward]
    I collegamenti tra unita' sono unidirezionali. 
    In questo caso la rete neurale \`e un grafo diretto aciclico. 
    Una rete neurale di tipo feed forward rappresenta una funzione dei suoi input nel senso che non ha uno stato interno o memoria. 
    Una rete feed forward \`e di solito divisa in \emph{layers} o livelli. 
    Ogni livello \`e un insieme di unita', ogni unita' riceve input solo da unita' del livello immediatamente precedente. 
    Una rete con un solo livello si dice anche \emph{single-layer} o \emph{monolivello}.
  \item[recurrent network]
    Ci possono essere dei cicli nei collegamenti. 
    Una rete di questo tipo rappresenta un sistema dinamico che pu\`o anche oscillare o avere un comportamento caotico. 
    Inoltre la riposta di una recurrent network dipende dallo stato iniziale della rete e dall'input quindi questo tipo di rete pu\`o supportare una memoria a breve termine.
\end{description}

