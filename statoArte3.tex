\begin{frame}
  \frametitle{Breath Analysis of Respiratory Flow using Tracheal Sounds}

%     In questo studio gli autori studiano le differenze che ci sono tra la fase inspiratoria e la fase espiratoria in due quantit\`a relative ad un segnale tracheale filtrato con un filtro passa banda. 
%     Queste due quantit\`a sono la media e la varianza logaritmica dell'energia. 
%     Lo studio usa uno spirometro per misurare il flusso. 
%     Il flusso viene diviso in base al valore assoluto in: basso, medio, alto e molto alto. 
%     Questo algoritmo quindi non ricava il flusso a partire dal suono ma \`e utile per comprendere e sviluppare algoritmi di riconoscimento delle fasi respiratorie.
% 
% 
%     \paragraph{input}
%       I dati presi in input sono delle registrazioni di suoni tracheali registrati su nove soggetti sani e non fumatori i quali non hanno mai avuto gravi malattie respiratorie. 
%       Inoltre lo studio aveva a disposizione anche il flusso d'aria registrato attraverso uno spirometro con pneumotacografo
%     \paragraph{algoritmo}
%       L'algoritmo ha due flussi di esecuzione indipendenti, il primo \`e il seguente:
%       \begin{enumerate}
% 	\item 
% 	  Filtro passa alto con frequenza di taglio di $70Hz$ per rimuovere il rumore a bassa frequenza.
% 	\item
% 	  Nelle fasi seguenti l'algoritmo considera solo le porzioni del suono registrate quando il segnale del flusso era al di sotto del $20\%$ del flusso medio o al di sopra del $20\%$ di esso.
% 	  Perch\'e in queste condizioni il suono tracheale si pu\`o considerare stazionario.
% 	\item
% 	  Lo spettro di potenza dei suoni della trachea \`e stato calcolato in una finestra di $50ms$ ($512$ campioni) con il $75\%$ di sovrapposizione tra finestre successive. 
% 	  Per ogni fase respiratoria durante la quale c'era un diverso flusso d'aria, \`e stata calcolata la media della potenza dei suoni tracheali in decibel  entro sei predefinite bande di frequenza: da $70$ a $300Hz$, da $300$ a $450Hz$, da $450$ a $600Hz$, da $600$ a $800Hz$, da $800$ a $1000Hz$ e da $1000$ a $1200Hz$. 
% 	\item
% 	  Dato che le intensit\`a dei suoni respiratori variano da soggetto a soggetto, per ogni soggetto i valori calcolati in precedenza sono stati normalizzati rispetto al valore massimo.
% 	\item
% 	  Si \`e poi calcolata la media dei valori normalizzati tra soggetti diversi per ogni fase respiratoria, inoltre sono state calcolate la media e l'errore standard per diversi tassi di flusso e intervalli di frequenza.
%       \end{enumerate} 
%       mentre il secondo flusso di esecuzione \`e:
%       \begin{enumerate}
% 	\item
% 	  Il segnale dei suoni della trachea sono stati filtrati attraverso un filtro passa alto nelle stesse frequenze menzionate in precedenza. 
% 	\item
% 	  Il segnale filtrato \`e stato in seguito segmentato in finestre di dimensione $50ms$ ($512$ campioni) con il $75\%$ di sovrapposizione tra finestre successive usando una finestra di Hanning. 
% 	\item
% 	  Si calcola il logaritmo della varianza dei segmenti precedenti.
% 	\item
% 	  In ciascuna finestra il valore precedente viene normalizzato rispetto al valore massimo per ridurre le interferenze dei suoni del cuore
% 	\item	
% 	  In seguito viene calcolata la media all'interno delle diverse bande di frequenza e diversi valori del flusso d'aria.
%       \end{enumerate}
%     
%     \paragraph{conclusioni}
%       % \begin{figure}
% 	%  \centering
% 	%  \includegraphics[width=0.9\textwidth]{../Dropbox/tesi respiro/articoli stato dell'arte 1/Breath Analysis of Respiratory Flow using Tracheal Sounds Figura.png}
% 	%  % Breath Analysis of Respiratory Flow using Tracheal Sounds Figura.png: 953x548 pixel, 72dpi, 33.62x19.33 cm, bb=0 0 953 548
% 	%   \label{baorfutsf}
% 	%   
%       % \end{figure}
%       % medium and high flow rate samples of the recorded flow signal along with the spectrogram of the corresponding tracheal sound signal, the normalized logvariance and the normalized Pave over [300-450] Hz.  
%       % It should be noted that the positive (negative) values of the recorded flow signal are related to the inspiratory (expiratory) phases of the respiration cycle. 
%       Da una analisi dello spettrogramma dei suoni tracheali, si pu\`o vedere che l'intensit\`a del suono tracheale aumenta con l'aumentare del valore assoluto del flusso. 
%       Inoltre anche la varianza logaritmica normalizzata e la media di potenza normalizzata seguono i cambiamenti nel valore assoluto del flusso. 
%       Nello spettrogramma, nella varianza logaritmica normalizzata e nella media normalizzata della potenza sono evidenti le transizioni di fase respiratoria. 
%       Quando il flusso era medio o alto, si ha una maggiore differenza di media dell'energia normalizzata tra la fase inspiratoria ed espiratoria nella banda di frequenze dai $300$ ai $450Hz$. 
%       Inoltre questa banda di frequenza ottiene la seconda maggior differenza di media dell'energia normalizzata tra la fase inspiratoria e la fase esipiratoria quando il flusso \`e basso o molto alto. 
%       Quindi questo intervallo di frequenza \`e stato scelto come ottimale per esaminare i cambiamenti nella media della potenza rispettivamente alle fasi respiratorie.
% 
%   
%   \subsection[Acoustical respiratory signal analysis and phase detection]{ \textit{Acoustical respiratory signal analysis and phase detection} \cite{ARSAPD}}
% 
%     Questo articolo propone un approccio di modellazione statistica per il riconoscimento delle fasi respiratorie.
%     \paragraph{input}
%       L'input usato per i test \`e preso da due segnali respiratori di due soggetti nell'intervallo di frequenza dai $250$ ai $312.5Hz$. 
%       Il primo dura $16s$ e contiene circa tre cicli di respirazione, il secondo dura $8$ secondi e contiene anch'esso tre cicli. 
%     \paragraph{algoritmo}
%       Prima di tutto l'analisi del suono viene fatta nel dominio dei pacchetti wavelet per incrementare l'accuratezza della determinazione dei suoni.
%       Il sistema di riconoscimento \`e implementato attraverso una segmentazione del segnale in tre parti alle quali viene associata una etichetta che pu\`o essere: 
%       inspirazione, espirazione e transizione. Viene adottata una rete Bayesiana, usando una versione con vincoli di una catena di Markov a triple. 
% %       Questo modello sfrutta un ciclo respiratorio a priori, con lo scopo di guidare l'algoritmo verso un accurato riconoscimento delle fasi della respirazione. 
%     \paragraph{conclusioni}
%       I risultati sono buoni ma \`e presente qualche errore di classificazione. Questo a causa del rumore ambientale nella prima registrazione e a causa della forte variazione di intensit\`a tra le fasi espiratorie nella seconda registrazione.
% 
%   \subsection[Computerized acoustical respiratory phase detection without airflow measurament]{\textit{Computerized acoustical respiratory phase detection without airflow measurament} \cite{CARPDWAM}}
% 
%     Questo studio sviluppa un metodo per riconoscere le fasi respiratorie a partire dai suoni prodotti dall'apparato respiratorio. Viene implementato anche un programma MATLAB.
%     \paragraph{input}
%       Sono stati studiati $21$ soggetti di et\`a dai $4$ ai $51$ anni. I soggetti sono stati divisi in gruppi di et\`a: $7$ bambini e $4$ bambine con una et\`a media di $10$ anni e $4$ uomini e $6$ donne con una et\`a media di $32$ anni. 
%       Tutti i soggetti godevano di buona salute e non avevano infezioni delle vie respiratorie nelle quattro settimane precedenti alla registrazione. 
%       Sei accelerometri sono stati usati per registrare i suoni respiratori. 
%       I dispositivi di registrazione erano attaccati sulla trachea e su varie parti del petto. 
%       Viene anche registrato in modo simultaneo, il flusso d'aria attraverso uno pneumotacografo con un trasduttore di differenza di pressione. 
%       L'algoritmo prende in input solo i suoni della registrazione mentre il flusso d'aria registrato dallo pneumotacografo serve esclusivamente per valutare la qualit\`a dei risultati dell'algoritmo. 
%       %     Airflow and breath sounds over the anterior chest and trachea were simultaneously recorded and stored in an IBM compatible personal computer. 
%       % In adult subjects, breath sounds were recorded at medium flow rate (15 ml/s/kg), while in children breath sounds were recorded at tidal (10 ml/s/kg), medium (15ml/s/kg) and high (35 ml/s/kg) flows. The subjects watched their airflow signal on an oscilloscope and were encouraged to reach the defined target flow in each experiment. 
%       % At each flow rate, a minimum of five complete breaths were recorded followed by a 5-second breath hold at the end of expiration to measure background noise.
%     \paragraph{algoritmo}
%       \begin{enumerate}
% 	\item
% 	  Il segnale \`e stato amplificato, filtrato attraverso un filtro passa banda con banda dai $50$ ai $200Hz$ e digitalizzato ad una frequenza di campionamento pari a $10240Hz$.
% 	\item 
% 	  Una finestra di Hanning \`e stata usata per segmentare il segnale sonoro in finestre di $2048$ campioni con il $50\%$ di sovrapposizione tra segmenti successivi.
% 	\item
% 	  Lo spettro di potenza di ogni segmento \`e stato calcolato con una trasformata veloce di Fourier. 
% 	  Le bande di frequenza usate per calcolare la potenza media di ogni segmento erano: dai $150$ ai $300Hz$, dai $300$ ai $450Hz$, dai $450$ ai $600Hz$ e dai $600$ ai $1200Hz$. 
% 	  L'algoritmo in modalit\`a di esecuzione semiautomatica offre all'utente la possibilit\`a di visualizzare i dati e di identificare eventuali suoni accidentali da rimuovere.
% 	\item
% 	  Per ogni segmento di $200ms$ viene calcolata la media della potenza del segnale registrato sulla parete del petto nella posizione con la quale si ottiene la maggiore differenza di suono tra inspirazione ed espirazione. 
% 	  Questo \`e risultato in un segnale nel quale i picchi corrispondono al massimo flusso d'aria durante l'inspirazione.
% 	\item
% 	  \`E stata usata una finestra mobile per riconoscere i picchi inspiratori per il segnale completo. 
% 	  Per venire in contro alla variabilit\`a nella frequenza respiratoria, la lunghezza della finestra \`e stata scelta in modo tale da approssimare la durata di un ciclo di respirazione e le finestra \`e stata spostata in incrementi che approssimano met\`a di un ciclo di respirazione. 
% 	  % MODIFICATO: 19/02/2013
% 	  % 	Dato che il suono della trachea \`e forte durante sia inspirazione che espirazione, \`e stato usato il segnale della trachea per trovare gli onset del respiro, mentre le fasi respiratorie sono state riconosciute grazie ai soli suoni del petto. 
% 	  % MODIFICATO: AGGIUNTO 19/02/2013
% 	  Dato che il suono polmonare \`e pi\`u forte durante l'inspirazione che durante l'espirazione, l'algoritmo usa i picchi del suono polmonare per determinare i picchi di inspirazione.
% 	  Invece i suoni tracheali sono forti sia durante l'inspirazione che durante l'espirazione e quindi li usa per determinare gli onset delle fasi respiratorie.
% 	\item
% 	  % MODIFICATO: AGGIUNTO 19/02/2013
% 	  Infine l'algoritmo classifica come inspirazione quella regione temporale che \`e compresa tra due onset tracheali e che contiene un picco polmonare. 
% 	  Tutto il resto viene classificato come espirazione.
%       \end{enumerate}
%     \paragraph{conclusioni}
%       Il software ha ottenuto una accuratezza massima nella stima delle fasi respiratorie senza l'uso dei dati sulla misurazione diretta del flusso d'aria.
% 
% 
% 
% \subsection[Respiratory onset detection using variance fractal dimension]{\textit{Respiratory onset detection using variance fractal dimension} \cite{RSDUVFD}} 
% 
%   Anche \cite{RSDUVFD} si occupa del problema di sviluppare un metodo acustico non invasivo per riconoscere le fasi respiratorie senza una misurazione diretta del flusso d'aria e lo fa usando la dimensione frattale. 
%   Questa \`e una misura della complessit\`a di un segnale. 
%   Una propriet\`a della dimensione frattale \`e che \`e indipendente dalla potenza del segnale. 
%   Questo sistema si basa sull'assunto che durante la transizione di fase respiratoria, il segnale sonoro ha un comportamento caotico a causa del cambiamento di momento nel flusso d'aria mentre questo cambia di direzione. 
%   Quindi, ipotizza che la varianza della dimensione frattale del suono respiratorio abbia un picco negli onset dei cicli respiratori.
%   \paragraph{input} 
%     Lo stesso input utilizzato dallo studio \cite{CARPDWAM}, il quale \`e gia stato descritto in precedenza.
%   \paragraph{algoritmo}
%     Per trovare gli onset dei cicli respiratori, viene calcolata la varianza della dimensione frattale usando un segmento di $128$ campioni pari a $12.5ms$ con il $50\%$ di sovrapposizione tra segmenti adiacenti. 
%     In seguito viene usata una finestra di lunghezza pari alla durata approssimativa della met\`a di un respiro cio\`e $0.7s$ con lo scopo di riconoscere i picchi nella varianza della dimensione frattale.
%   \paragraph{conclusioni}
%      Da un confronto tra il riconoscimento degli onset con il reale flusso d'aria i risultati mostrano che l'intervallo di errore va dai $31ms$ ai $49ms$. L'aspetto positivo di questo algoritmo \`e che fa una analisi esclusivamente nel dominio del tempo del segnale.
% 
% 
% \subsection[Automated respiratory phase detection by acoustical means]{\textit{Automated respiratory phase detection by acoustical means} \cite{DECE}}
% 
% 
%   Lo studio \cite{DECE} si concentra sull'automazione del processo di riconoscimento per via acustica delle fasi della respirazione senza l'ausilio di misure del flusso. 
%   Viene implementato un programma in C++ dotato di una interfaccia grafica in grado di lavorare in modo semiautomatico o completamente automatizzato.
%   \paragraph{input}
%     L'input \`e costituito da un insieme di $17$ suoni registrati sulla trachea e sul torace (sul secondo interspazio sinistro medioclavicolare e sul terzo 
%     interspazio destro medioclavicolare) di $11$ soggetti sani con et\`a dai $4$ ai $35$ anni. 
%     Inoltre erano disponibili anche i dati del flusso d'aria misurati 
%     attraverso uno pneumotacografo.
%   \paragraph{algoritmo}
%     \begin{enumerate}
%       \item 
% 	Il segnale del suono della respirazione viene segmentato in segmenti di lunghezza $100ms$, ognuno con una sovrapposizione del $50\%$ tra segmenti adiacenti. 
%       \item
% 	Per ogni segmento viene calcolato lo spettro di potenza usando una trasformata veloce di Fourier.
%       \item
% 	I segnali del torace vengono filtrati lasciando una banda di frequenza dai $150Hz$ ai $300Hz$ mentre i segnali della trachea vengono filtrati lasciando una banda di frequenza dai $150Hz$ ai $600Hz$
%       \item
% 	Viene presa la potenza media dei segnali.
%       \item
% 	Per riconoscere i picchi di inspirazione dai segnali del petto, inizialmente, viene calcolata la pendenza per ogni campione, prendendo la differenza tra punti adiacenti. 
% 	I campioni che hanno una differenza positiva con il campione precedente e una differenza negativa con il campione successivo sono i possibili picchi.
%       \item
% 	Per affinare la ricerca dei picchi inspiratori, viene usata una piccola finestra mobile. 
% 	Dato che il valore del flusso pu\`o variare molto da respiro a respiro, la finestra \`e stata scelta in modo da coprire un ciclo completo di respirazione nel segnale (approssimativamente $2s$ quindi $20$ campioni). Una finestra di questo tipo \`e applicata ai picchi determinati al punto precedente. Vengono determinati i punti massimi della finestra che hanno almeno una deviazione standard al di sopra della media. 
% 	Ci si aspetta che i picchi di inspirazione siano molto pi\`u alti dei picchi di espirazione per lo spettro di potenza dei suoni registrati sul petto. 
% 	Poich\'e i possibili picchi possono consistere sia in picchi di inspirazione che in picchi di di espirazione, bisogna usare una soglia per eliminare i  picchi di espirazione. 
% 	L'algoritmo di riconoscimento dei picchi \`e progettato per essere usato sia in modalit\`a automatica che in modalit\`a semiautomatica. 
% 	Nella modalit\`a semiautomatica, gli utenti possono aggiungere i picchi mancanti o eliminare i picchi riconosciuti dopo aver dato un'occhiata alla potenza media dello spettro dei segnali.
%       \item
% 	Per trovare gli onset del respiro dai segnali tracheali, prima, si calcola la pendenza ad ogni campione. 
%       \item
% 	In questo caso, i campioni che hanno una differenza negativa rispetto ai campioni precedenti e una differenza positiva rispetto ai campioni successivi vengono selezionati come potenziali onset. 
% 	Dato che in media la durata delle fasi respiratorie \`e approssimativamente $1s$, la distanza tra un picco e l'onset pi\`u vicino \`e di circa $500ms$ o $5$ campioni. 
% 	Per stare sicuri, viene considerata troppo vicina una differenza di $2$ campioni($200ms$), dunque i potenziali onset riconosciuti che sono considerati troppo vicini ad uno dei picchi precedenti vengono esclusi dalla selezione. 
%       \item
% 	Per calcolare il valore medio e la deviazione standard di ogni campione, si usa una finestra mobile di lunga $10$ campioni(approssimativamente met\`a della durata di un respiro). 
% 	I potenziali onset con deviazione standard almeno $0.5$ pi\`u piccola della media locale vengono scelti per una successiva analisi. 
%       \item
% 	Nel processo di riconoscimento dei picchi tramite i risultati ottenuti viene stimata la distanza media tra due picchi inspiratori.
% 	La lunghezza media di una fase respiratoria \`e calcolata come la met\`a della media della distanza tra i picchi. 
% 	Questa informazione \`e usata per raggruppare i potenziali onset in cluster.
% 	Si classifica come onset del respiro solo il punto minimo all'interno di ogni cluster.
%       \item
% 	Per massimizzare l'accuratezza del programma di riconoscimento  della fase respiratoria viene fatta un ulteriore ottimizzazione aggiungendo un algoritmo di autocorrezione.
% 	Questo algoritmo confronta gli onset trovati con i picchi stimati per assicurarsi che esistano solo due onset tra due picchi inspiratori. 
% 	Gli onset che non vengono riconosciuti durante il processo di selezione precedente vengono aggiunti mentre gli onset extra che esistono tra 
% 	due picchi vengono rimossi. 
% 	Con i picchi stimati e gli onset si possono prevedere le fasi respiratorie o la direzione del flusso d'aria. 
% 	Gli intervalli tra due onset con un picco sono inspiratori mentre gli intervalli senza picchi tra gli onset sono espiratori.
%     \end{enumerate}
%     
%   \paragraph{conclusioni}
%     Per calcolare le prestazioni complessive del sistema sono stati valutati tre parametri:
%     \begin{itemize}
%       \item 
% 	Percentuale delle fasi respiratorie riconosciute in modo corretto dall'algoritmo di riconoscimento dei picchi.
%       \item
% 	Percentuale degli onset riconosciuti in modo corretto dall'algoritmo di riconoscimento degli onset.
%       \item
% 	Media delle differenze in millisecondi tra gli onset riconosciuti e quelli trovati direttamente dalla misurazione del flusso.
%     \end{itemize}
%      %[immagine tabella risultati]
%     Nella modalit\`a semiautomatica i picchi di inspirazione vengono riconosciuti con una accuratezza del $100\%$. 
%     In modalit\`a completamente automatica l'accuratezza media \`e del $93\%$ con una deviazione standard del $7\%$. 
%     Confrontato con il flusso misurato direttamente, in media, c'\`e solo un ritardo di $118 ms$ di riconoscimento tra gli onset rilevati e quelli reali.
%     Dato che \`e inevitabile un minimo flusso critico all'interno della trachea per generare un suono udibile dallo stetoscopio, ci si aspetta una differenza tra gli onset reali dei cicli della respirazione ottenuti attraverso uno pneumotacografo e gli onset stimati dal sistema. 
%     Studi precedenti mostrano che questa differenza \`e di circa $40ms$\cite{CARPDWAM}. 
%     Questo sistema stima gli onset delle fasi respiratorie e la direzione del flusso d'aria ma non tenta di stimare anche il valore assoluto del flusso.
% 
% 
% 
% 
% 
% 
% % \subsection{\cite{SPMNIRM}}
% % 
% % \cite{SPMNIRM} investiga la fattibilit\`a di un monitoraggio affidabile della respirazione attraverso sensori non invasivi. 
% %   \paragraph{input}
% %     Il lavoro \`e stato testato su due database. Uno di questi \`e parte dell'archivio Physionet. 
% % I segnali presi in considerazione sono: elettrocardiogramma(ECG), segnale di impedenza pletismografico transtoracico(IP), segnale fotopletismografico(PPG). 
% % Questi meccanismi di monitoraggio sono non invasivi e disponibili in ambienti clinici. 
% % I segnali considerati sono:
% % \begin{description}
% % \item[Elettrocardiogramma]
% %   Ci sono tre caratteristiche di un ECG le cui cause sono funzionalmente correlate alla respirazione:
% %   \begin{description}
% %     \item[Respiratory Sinus Arrhythmia] 
% %       Si riferisce alla variazione ciclica  nella frequenza cardiaca associata alla respirazione.
% %  La frequenza cardiaca accelera durante l'inspirazione e decelera durante l'espirazione. 
% % Il valore assoluto dell'accelerazione varia da persona a persona e decresce rispetto alla frequenza respiratoria.
% %     \item[Modulazione di ampiezza R-S durante l'inspirazione] 
% % %       L'apice del cuore \`e allungato verso l'addome a causa del riempimento dei polmoni, aiutati dallo spostamento verso il basso del diaframma. 
% % Durante l'espirazione, l'elevazione del diaframma comprime l'apice del cuore verso il petto. 
% %  Dunque la respirazione cambia l'angolo che i vettori cardiaci fanno rispetto ad un vettore di riferimento. 
% % Questo cambiamento modifica l'ampiezza del segnale ECD. Si nota che la modulazione dell'ampiezza QRS \`e particolarmente significativa. 
% %     \item[Baseline Wander] 
% %       Low frequency wander del segnale ECG possono essere causate dalla respirazione. 
% % L'espansione e la contrazione del petto che accompagna la respirazione risulta in un movimento degli elettrodi del petto rispettivamente al cuore.
% %  Questo pu\`o causare una baseline wander nell'ECG di solito vista nel respiro profondo o esagerato.
% %   \end{description}
% %   \item[Pressione sanguigna]
% % %     La pressione sanguigna \`e la pressione del sangue che fluisce nei vasi sanguigi contro le pareti degli stessi. 
% % Dipende dalla quantit\`a del flusso di sangue e dalla resistenza delle pareti dei vasi sanguigni. 
% % Ogni volta che il cuore batte, aumenta il sangue pompato nelle arterie.
% %  Questo aumenta la pressione nelle arterie. 
% % Tra due battiti cardiaci contigui la pressione arteriosa diminuisce. 
% % La pressione sanguigna si misura con due quantit\`a. 
% % La prima quantit\`a, detta sistolica, \`e la pressione del sangue contro le pareti arteriose quando il cuore si contrae.
% %  La seconda, detta diastolica, \`e la pressione del sangue contro le pareti arteriose quando il cuore si rilassa tra due battiti contigui.
% %  Ci sono tre caratteristiche della pressione sanguigna le cui cause sono funzionalmente correlate alla respirazione:
% %     \begin{description}
% %       \item[Pulsus Paradoxus]
% % 	\`E il decremento nella pressione sanguigna sistolica che \`e proporzionale ai cambiamenti nella pressione intratoracica durante l'inspirazione e l'espirazione. 
% %     \item[Blood Pressure Variability]
% %       L'intevallo di tempo che intercorre tra due picchi nella pressione sistolica sono legati alla variazione ciclica legata alla respirazione.
% % \end{description}
% %   \item[Photoplethysmography]
% %     La fotopletismografia \`e una tecina optoelettrica per misurare le onde cardiovascolari attraverso il corpo. 
% % Queste onde sono causate dalle pulsazioni periodiche nel volume del sangue arterioso. 
% % Si misurano attraverso il conseguente cambiamento nell'assorbimento ottico. 
% % Il sistema di misura consiste in una sorgente di luce di solito infrarossa, un rilevatore e un sistema di rilevamento processamento e visualizzazione del segname. 
% % La luce infrarossa viene assorbita bene dal sangue e poco dai tessuti. 
% % Quindi i cambiamenti di volume nel sangue vengono osservati con un contrasto ragionevole.
% %     
% % %     PPG measures the pulse wave caused by periodi    In questo studio gli autori studiano le differenze che ci sono tra la fase inspiratoria e la fase espiratoria in due quantit\`a relative ad un segnale tracheale filtrato con un filtro passa banda. 
    Queste due quantit\`a sono la media e la varianza logaritmica dell'energia. 
    Lo studio usa uno spirometro per misurare il flusso. 
    Il flusso viene diviso in base al valore assoluto in: basso, medio, alto e molto alto. 
    Questo algoritmo quindi non ricava il flusso a partire dal suono ma \`e utile per comprendere e sviluppare algoritmi di riconoscimento delle fasi respiratorie.


    \paragraph{input}
      I dati presi in input sono delle registrazioni di suoni tracheali registrati su nove soggetti sani e non fumatori i quali non hanno mai avuto gravi malattie respiratorie. 
      Inoltre lo studio aveva a disposizione anche il flusso d'aria registrato attraverso uno spirometro con pneumotacografo
    \paragraph{algoritmo}
      L'algoritmo ha due flussi di esecuzione indipendenti, il primo \`e il seguente:
      \begin{enumerate}
	\item 
	  Filtro passa alto con frequenza di taglio di $70Hz$ per rimuovere il rumore a bassa frequenza.
	\item
	  Nelle fasi seguenti l'algoritmo considera solo le porzioni del suono registrate quando il segnale del flusso era al di sotto del $20\%$ del flusso medio o al di sopra del $20\%$ di esso.
	  Perch\'e in queste condizioni il suono tracheale si pu\`o considerare stazionario.
	\item
	  Lo spettro di potenza dei suoni della trachea \`e stato calcolato in una finestra di $50ms$ ($512$ campioni) con il $75\%$ di sovrapposizione tra finestre successive. 
	  Per ogni fase respiratoria durante la quale c'era un diverso flusso d'aria, \`e stata calcolata la media della potenza dei suoni tracheali in decibel  entro sei predefinite bande di frequenza: da $70$ a $300Hz$, da $300$ a $450Hz$, da $450$ a $600Hz$, da $600$ a $800Hz$, da $800$ a $1000Hz$ e da $1000$ a $1200Hz$. 
	\item
	  Dato che le intensit\`a dei suoni respiratori variano da soggetto a soggetto, per ogni soggetto i valori calcolati in precedenza sono stati normalizzati rispetto al valore massimo.
	\item
	  Si \`e poi calcolata la media dei valori normalizzati tra soggetti diversi per ogni fase respiratoria, inoltre sono state calcolate la media e l'errore standard per diversi tassi di flusso e intervalli di frequenza.
      \end{enumerate} 
      mentre il secondo flusso di esecuzione \`e:
      \begin{enumerate}
	\item
	  Il segnale dei suoni della trachea sono stati filtrati attraverso un filtro passa alto nelle stesse frequenze menzionate in precedenza. 
	\item
	  Il segnale filtrato \`e stato in seguito segmentato in finestre di dimensione $50ms$ ($512$ campioni) con il $75\%$ di sovrapposizione tra finestre successive usando una finestra di Hanning. 
	\item
	  Si calcola il logaritmo della varianza dei segmenti precedenti.
	\item
	  In ciascuna finestra il valore precedente viene normalizzato rispetto al valore massimo per ridurre le interferenze dei suoni del cuore
	\item	
	  In seguito viene calcolata la media all'interno delle diverse bande di frequenza e diversi valori del flusso d'aria.
      \end{enumerate}
    
    \paragraph{conclusioni}
      % \begin{figure}
	%  \centering
	%  \includegraphics[width=0.9\textwidth]{../Dropbox/tesi respiro/articoli stato dell'arte 1/Breath Analysis of Respiratory Flow using Tracheal Sounds Figura.png}
	%  % Breath Analysis of Respiratory Flow using Tracheal Sounds Figura.png: 953x548 pixel, 72dpi, 33.62x19.33 cm, bb=0 0 953 548
	%   \label{baorfutsf}
	%   
      % \end{figure}
      % medium and high flow rate samples of the recorded flow signal along with the spectrogram of the corresponding tracheal sound signal, the normalized logvariance and the normalized Pave over [300-450] Hz.  
      % It should be noted that the positive (negative) values of the recorded flow signal are related to the inspiratory (expiratory) phases of the respiration cycle. 
      Da una analisi dello spettrogramma dei suoni tracheali, si pu\`o vedere che l'intensit\`a del suono tracheale aumenta con l'aumentare del valore assoluto del flusso. 
      Inoltre anche la varianza logaritmica normalizzata e la media di potenza normalizzata seguono i cambiamenti nel valore assoluto del flusso. 
      Nello spettrogramma, nella varianza logaritmica normalizzata e nella media normalizzata della potenza sono evidenti le transizioni di fase respiratoria. 
      Quando il flusso era medio o alto, si ha una maggiore differenza di media dell'energia normalizzata tra la fase inspiratoria ed espiratoria nella banda di frequenze dai $300$ ai $450Hz$. 
      Inoltre questa banda di frequenza ottiene la seconda maggior differenza di media dell'energia normalizzata tra la fase inspiratoria e la fase esipiratoria quando il flusso \`e basso o molto alto. 
      Quindi questo intervallo di frequenza \`e stato scelto come ottimale per esaminare i cambiamenti nella media della potenza rispettivamente alle fasi respiratorie.

  
  \subsection[Acoustical respiratory signal analysis and phase detection]{ \textit{Acoustical respiratory signal analysis and phase detection} \cite{ARSAPD}}

    Questo articolo propone un approccio di modellazione statistica per il riconoscimento delle fasi respiratorie.
    \paragraph{input}
      L'input usato per i test \`e preso da due segnali respiratori di due soggetti nell'intervallo di frequenza dai $250$ ai $312.5Hz$. 
      Il primo dura $16s$ e contiene circa tre cicli di respirazione, il secondo dura $8$ secondi e contiene anch'esso tre cicli. 
    \paragraph{algoritmo}
      Prima di tutto l'analisi del suono viene fatta nel dominio dei pacchetti wavelet per incrementare l'accuratezza della determinazione dei suoni.
      Il sistema di riconoscimento \`e implementato attraverso una segmentazione del segnale in tre parti alle quali viene associata una etichetta che pu\`o essere: 
      inspirazione, espirazione e transizione. Viene adottata una rete Bayesiana, usando una versione con vincoli di una catena di Markov a triple. 
%       Questo modello sfrutta un ciclo respiratorio a priori, con lo scopo di guidare l'algoritmo verso un accurato riconoscimento delle fasi della respirazione. 
    \paragraph{conclusioni}
      I risultati sono buoni ma \`e presente qualche errore di classificazione. Questo a causa del rumore ambientale nella prima registrazione e a causa della forte variazione di intensit\`a tra le fasi espiratorie nella seconda registrazione.

  \subsection[Computerized acoustical respiratory phase detection without airflow measurament]{\textit{Computerized acoustical respiratory phase detection without airflow measurament} \cite{CARPDWAM}}

    Questo studio sviluppa un metodo per riconoscere le fasi respiratorie a partire dai suoni prodotti dall'apparato respiratorio. Viene implementato anche un programma MATLAB.
    \paragraph{input}
      Sono stati studiati $21$ soggetti di et\`a dai $4$ ai $51$ anni. I soggetti sono stati divisi in gruppi di et\`a: $7$ bambini e $4$ bambine con una et\`a media di $10$ anni e $4$ uomini e $6$ donne con una et\`a media di $32$ anni. 
      Tutti i soggetti godevano di buona salute e non avevano infezioni delle vie respiratorie nelle quattro settimane precedenti alla registrazione. 
      Sei accelerometri sono stati usati per registrare i suoni respiratori. 
      I dispositivi di registrazione erano attaccati sulla trachea e su varie parti del petto. 
      Viene anche registrato in modo simultaneo, il flusso d'aria attraverso uno pneumotacografo con un trasduttore di differenza di pressione. 
      L'algoritmo prende in input solo i suoni della registrazione mentre il flusso d'aria registrato dallo pneumotacografo serve esclusivamente per valutare la qualit\`a dei risultati dell'algoritmo. 
      %     Airflow and breath sounds over the anterior chest and trachea were simultaneously recorded and stored in an IBM compatible personal computer. 
      % In adult subjects, breath sounds were recorded at medium flow rate (15 ml/s/kg), while in children breath sounds were recorded at tidal (10 ml/s/kg), medium (15ml/s/kg) and high (35 ml/s/kg) flows. The subjects watched their airflow signal on an oscilloscope and were encouraged to reach the defined target flow in each experiment. 
      % At each flow rate, a minimum of five complete breaths were recorded followed by a 5-second breath hold at the end of expiration to measure background noise.
    \paragraph{algoritmo}
      \begin{enumerate}
	\item
	  Il segnale \`e stato amplificato, filtrato attraverso un filtro passa banda con banda dai $50$ ai $200Hz$ e digitalizzato ad una frequenza di campionamento pari a $10240Hz$.
	\item 
	  Una finestra di Hanning \`e stata usata per segmentare il segnale sonoro in finestre di $2048$ campioni con il $50\%$ di sovrapposizione tra segmenti successivi.
	\item
	  Lo spettro di potenza di ogni segmento \`e stato calcolato con una trasformata veloce di Fourier. 
	  Le bande di frequenza usate per calcolare la potenza media di ogni segmento erano: dai $150$ ai $300Hz$, dai $300$ ai $450Hz$, dai $450$ ai $600Hz$ e dai $600$ ai $1200Hz$. 
	  L'algoritmo in modalit\`a di esecuzione semiautomatica offre all'utente la possibilit\`a di visualizzare i dati e di identificare eventuali suoni accidentali da rimuovere.
	\item
	  Per ogni segmento di $200ms$ viene calcolata la media della potenza del segnale registrato sulla parete del petto nella posizione con la quale si ottiene la maggiore differenza di suono tra inspirazione ed espirazione. 
	  Questo \`e risultato in un segnale nel quale i picchi corrispondono al massimo flusso d'aria durante l'inspirazione.
	\item
	  \`E stata usata una finestra mobile per riconoscere i picchi inspiratori per il segnale completo. 
	  Per venire in contro alla variabilit\`a nella frequenza respiratoria, la lunghezza della finestra \`e stata scelta in modo tale da approssimare la durata di un ciclo di respirazione e le finestra \`e stata spostata in incrementi che approssimano met\`a di un ciclo di respirazione. 
	  % MODIFICATO: 19/02/2013
	  % 	Dato che il suono della trachea \`e forte durante sia inspirazione che espirazione, \`e stato usato il segnale della trachea per trovare gli onset del respiro, mentre le fasi respiratorie sono state riconosciute grazie ai soli suoni del petto. 
	  % MODIFICATO: AGGIUNTO 19/02/2013
	  Dato che il suono polmonare \`e pi\`u forte durante l'inspirazione che durante l'espirazione, l'algoritmo usa i picchi del suono polmonare per determinare i picchi di inspirazione.
	  Invece i suoni tracheali sono forti sia durante l'inspirazione che durante l'espirazione e quindi li usa per determinare gli onset delle fasi respiratorie.
	\item
	  % MODIFICATO: AGGIUNTO 19/02/2013
	  Infine l'algoritmo classifica come inspirazione quella regione temporale che \`e compresa tra due onset tracheali e che contiene un picco polmonare. 
	  Tutto il resto viene classificato come espirazione.
      \end{enumerate}
    \paragraph{conclusioni}
      Il software ha ottenuto una accuratezza massima nella stima delle fasi respiratorie senza l'uso dei dati sulla misurazione diretta del flusso d'aria.



\subsection[Respiratory onset detection using variance fractal dimension]{\textit{Respiratory onset detection using variance fractal dimension} \cite{RSDUVFD}} 

  Anche \cite{RSDUVFD} si occupa del problema di sviluppare un metodo acustico non invasivo per riconoscere le fasi respiratorie senza una misurazione diretta del flusso d'aria e lo fa usando la dimensione frattale. 
  Questa \`e una misura della complessit\`a di un segnale. 
  Una propriet\`a della dimensione frattale \`e che \`e indipendente dalla potenza del segnale. 
  Questo sistema si basa sull'assunto che durante la transizione di fase respiratoria, il segnale sonoro ha un comportamento caotico a causa del cambiamento di momento nel flusso d'aria mentre questo cambia di direzione. 
  Quindi, ipotizza che la varianza della dimensione frattale del suono respiratorio abbia un picco negli onset dei cicli respiratori.
  \paragraph{input} 
    Lo stesso input utilizzato dallo studio \cite{CARPDWAM}, il quale \`e gia stato descritto in precedenza.
  \paragraph{algoritmo}
    Per trovare gli onset dei cicli respiratori, viene calcolata la varianza della dimensione frattale usando un segmento di $128$ campioni pari a $12.5ms$ con il $50\%$ di sovrapposizione tra segmenti adiacenti. 
    In seguito viene usata una finestra di lunghezza pari alla durata approssimativa della met\`a di un respiro cio\`e $0.7s$ con lo scopo di riconoscere i picchi nella varianza della dimensione frattale.
  \paragraph{conclusioni}
     Da un confronto tra il riconoscimento degli onset con il reale flusso d'aria i risultati mostrano che l'intervallo di errore va dai $31ms$ ai $49ms$. L'aspetto positivo di questo algoritmo \`e che fa una analisi esclusivamente nel dominio del tempo del segnale.


\subsection[Automated respiratory phase detection by acoustical means]{\textit{Automated respiratory phase detection by acoustical means} \cite{DECE}}


  Lo studio \cite{DECE} si concentra sull'automazione del processo di riconoscimento per via acustica delle fasi della respirazione senza l'ausilio di misure del flusso. 
  Viene implementato un programma in C++ dotato di una interfaccia grafica in grado di lavorare in modo semiautomatico o completamente automatizzato.
  \paragraph{input}
    L'input \`e costituito da un insieme di $17$ suoni registrati sulla trachea e sul torace (sul secondo interspazio sinistro medioclavicolare e sul terzo 
    interspazio destro medioclavicolare) di $11$ soggetti sani con et\`a dai $4$ ai $35$ anni. 
    Inoltre erano disponibili anche i dati del flusso d'aria misurati 
    attraverso uno pneumotacografo.
  \paragraph{algoritmo}
    \begin{enumerate}
      \item 
	Il segnale del suono della respirazione viene segmentato in segmenti di lunghezza $100ms$, ognuno con una sovrapposizione del $50\%$ tra segmenti adiacenti. 
      \item
	Per ogni segmento viene calcolato lo spettro di potenza usando una trasformata veloce di Fourier.
      \item
	I segnali del torace vengono filtrati lasciando una banda di frequenza dai $150Hz$ ai $300Hz$ mentre i segnali della trachea vengono filtrati lasciando una banda di frequenza dai $150Hz$ ai $600Hz$
      \item
	Viene presa la potenza media dei segnali.
      \item
	Per riconoscere i picchi di inspirazione dai segnali del petto, inizialmente, viene calcolata la pendenza per ogni campione, prendendo la differenza tra punti adiacenti. 
	I campioni che hanno una differenza positiva con il campione precedente e una differenza negativa con il campione successivo sono i possibili picchi.
      \item
	Per affinare la ricerca dei picchi inspiratori, viene usata una piccola finestra mobile. 
	Dato che il valore del flusso pu\`o variare molto da respiro a respiro, la finestra \`e stata scelta in modo da coprire un ciclo completo di respirazione nel segnale (approssimativamente $2s$ quindi $20$ campioni). Una finestra di questo tipo \`e applicata ai picchi determinati al punto precedente. Vengono determinati i punti massimi della finestra che hanno almeno una deviazione standard al di sopra della media. 
	Ci si aspetta che i picchi di inspirazione siano molto pi\`u alti dei picchi di espirazione per lo spettro di potenza dei suoni registrati sul petto. 
	Poich\'e i possibili picchi possono consistere sia in picchi di inspirazione che in picchi di di espirazione, bisogna usare una soglia per eliminare i  picchi di espirazione. 
	L'algoritmo di riconoscimento dei picchi \`e progettato per essere usato sia in modalit\`a automatica che in modalit\`a semiautomatica. 
	Nella modalit\`a semiautomatica, gli utenti possono aggiungere i picchi mancanti o eliminare i picchi riconosciuti dopo aver dato un'occhiata alla potenza media dello spettro dei segnali.
      \item
	Per trovare gli onset del respiro dai segnali tracheali, prima, si calcola la pendenza ad ogni campione. 
      \item
	In questo caso, i campioni che hanno una differenza negativa rispetto ai campioni precedenti e una differenza positiva rispetto ai campioni successivi vengono selezionati come potenziali onset. 
	Dato che in media la durata delle fasi respiratorie \`e approssimativamente $1s$, la distanza tra un picco e l'onset pi\`u vicino \`e di circa $500ms$ o $5$ campioni. 
	Per stare sicuri, viene considerata troppo vicina una differenza di $2$ campioni($200ms$), dunque i potenziali onset riconosciuti che sono considerati troppo vicini ad uno dei picchi precedenti vengono esclusi dalla selezione. 
      \item
	Per calcolare il valore medio e la deviazione standard di ogni campione, si usa una finestra mobile di lunga $10$ campioni(approssimativamente met\`a della durata di un respiro). 
	I potenziali onset con deviazione standard almeno $0.5$ pi\`u piccola della media locale vengono scelti per una successiva analisi. 
      \item
	Nel processo di riconoscimento dei picchi tramite i risultati ottenuti viene stimata la distanza media tra due picchi inspiratori.
	La lunghezza media di una fase respiratoria \`e calcolata come la met\`a della media della distanza tra i picchi. 
	Questa informazione \`e usata per raggruppare i potenziali onset in cluster.
	Si classifica come onset del respiro solo il punto minimo all'interno di ogni cluster.
      \item
	Per massimizzare l'accuratezza del programma di riconoscimento  della fase respiratoria viene fatta un ulteriore ottimizzazione aggiungendo un algoritmo di autocorrezione.
	Questo algoritmo confronta gli onset trovati con i picchi stimati per assicurarsi che esistano solo due onset tra due picchi inspiratori. 
	Gli onset che non vengono riconosciuti durante il processo di selezione precedente vengono aggiunti mentre gli onset extra che esistono tra 
	due picchi vengono rimossi. 
	Con i picchi stimati e gli onset si possono prevedere le fasi respiratorie o la direzione del flusso d'aria. 
	Gli intervalli tra due onset con un picco sono inspiratori mentre gli intervalli senza picchi tra gli onset sono espiratori.
    \end{enumerate}
    
  \paragraph{conclusioni}
    Per calcolare le prestazioni complessive del sistema sono stati valutati tre parametri:
    \begin{itemize}
      \item 
	Percentuale delle fasi respiratorie riconosciute in modo corretto dall'algoritmo di riconoscimento dei picchi.
      \item
	Percentuale degli onset riconosciuti in modo corretto dall'algoritmo di riconoscimento degli onset.
      \item
	Media delle differenze in millisecondi tra gli onset riconosciuti e quelli trovati direttamente dalla misurazione del flusso.
    \end{itemize}
     %[immagine tabella risultati]
    Nella modalit\`a semiautomatica i picchi di inspirazione vengono riconosciuti con una accuratezza del $100\%$. 
    In modalit\`a completamente automatica l'accuratezza media \`e del $93\%$ con una deviazione standard del $7\%$. 
    Confrontato con il flusso misurato direttamente, in media, c'\`e solo un ritardo di $118 ms$ di riconoscimento tra gli onset rilevati e quelli reali.
    Dato che \`e inevitabile un minimo flusso critico all'interno della trachea per generare un suono udibile dallo stetoscopio, ci si aspetta una differenza tra gli onset reali dei cicli della respirazione ottenuti attraverso uno pneumotacografo e gli onset stimati dal sistema. 
    Studi precedenti mostrano che questa differenza \`e di circa $40ms$\cite{CARPDWAM}. 
    Questo sistema stima gli onset delle fasi respiratorie e la direzione del flusso d'aria ma non tenta di stimare anche il valore assoluto del flusso.






% \subsection{\cite{SPMNIRM}}
% 
% \cite{SPMNIRM} investiga la fattibilit\`a di un monitoraggio affidabile della respirazione attraverso sensori non invasivi. 
%   \paragraph{input}
%     Il lavoro \`e stato testato su due database. Uno di questi \`e parte dell'archivio Physionet. 
% I segnali presi in considerazione sono: elettrocardiogramma(ECG), segnale di impedenza pletismografico transtoracico(IP), segnale fotopletismografico(PPG). 
% Questi meccanismi di monitoraggio sono non invasivi e disponibili in ambienti clinici. 
% I segnali considerati sono:
% \begin{description}
% \item[Elettrocardiogramma]
%   Ci sono tre caratteristiche di un ECG le cui cause sono funzionalmente correlate alla respirazione:
%   \begin{description}
%     \item[Respiratory Sinus Arrhythmia] 
%       Si riferisce alla variazione ciclica  nella frequenza cardiaca associata alla respirazione.
%  La frequenza cardiaca accelera durante l'inspirazione e decelera durante l'espirazione. 
% Il valore assoluto dell'accelerazione varia da persona a persona e decresce rispetto alla frequenza respiratoria.
%     \item[Modulazione di ampiezza R-S durante l'inspirazione] 
% %       L'apice del cuore \`e allungato verso l'addome a causa del riempimento dei polmoni, aiutati dallo spostamento verso il basso del diaframma. 
% Durante l'espirazione, l'elevazione del diaframma comprime l'apice del cuore verso il petto. 
%  Dunque la respirazione cambia l'angolo che i vettori cardiaci fanno rispetto ad un vettore di riferimento. 
% Questo cambiamento modifica l'ampiezza del segnale ECD. Si nota che la modulazione dell'ampiezza QRS \`e particolarmente significativa. 
%     \item[Baseline Wander] 
%       Low frequency wander del segnale ECG possono essere causate dalla respirazione. 
% L'espansione e la contrazione del petto che accompagna la respirazione risulta in un movimento degli elettrodi del petto rispettivamente al cuore.
%  Questo pu\`o causare una baseline wander nell'ECG di solito vista nel respiro profondo o esagerato.
%   \end{description}
%   \item[Pressione sanguigna]
% %     La pressione sanguigna \`e la pressione del sangue che fluisce nei vasi sanguigi contro le pareti degli stessi. 
% Dipende dalla quantit\`a del flusso di sangue e dalla resistenza delle pareti dei vasi sanguigni. 
% Ogni volta che il cuore batte, aumenta il sangue pompato nelle arterie.
%  Questo aumenta la pressione nelle arterie. 
% Tra due battiti cardiaci contigui la pressione arteriosa diminuisce. 
% La pressione sanguigna si misura con due quantit\`a. 
% La prima quantit\`a, detta sistolica, \`e la pressione del sangue contro le pareti arteriose quando il cuore si contrae.
%  La seconda, detta diastolica, \`e la pressione del sangue contro le pareti arteriose quando il cuore si rilassa tra due battiti contigui.
%  Ci sono tre caratteristiche della pressione sanguigna le cui cause sono funzionalmente correlate alla respirazione:
%     \begin{description}
%       \item[Pulsus Paradoxus]
% 	\`E il decremento nella pressione sanguigna sistolica che \`e proporzionale ai cambiamenti nella pressione intratoracica durante l'inspirazione e l'espirazione. 
%     \item[Blood Pressure Variability]
%       L'intevallo di tempo che intercorre tra due picchi nella pressione sistolica sono legati alla variazione ciclica legata alla respirazione.
% \end{description}
%   \item[Photoplethysmography]
%     La fotopletismografia \`e una tecina optoelettrica per misurare le onde cardiovascolari attraverso il corpo. 
% Queste onde sono causate dalle pulsazioni periodiche nel volume del sangue arterioso. 
% Si misurano attraverso il conseguente cambiamento nell'assorbimento ottico. 
% Il sistema di misura consiste in una sorgente di luce di solito infrarossa, un rilevatore e un sistema di rilevamento processamento e visualizzazione del segname. 
% La luce infrarossa viene assorbita bene dal sangue e poco dai tessuti. 
% Quindi i cambiamenti di volume nel sangue vengono osservati con un contrasto ragionevole.
%     
% %     PPG measures the pulse wave caused by periodic pulsations in arterial blood volume.
%  These periodic pulsations give rise to the blood pressure waveform. 
% Several studies have attempted to determine the correlation between the respiration induced effects in the blood pressure and PPG signals, the aim of these studies being to investigate the possibility of non-invasive analysis of pulsus paradoxus using PPG. 
% Steele et al. [97] report on different methods of continuous non-invasive monitoring of pulsus paradoxus.
%  The reference measurement of pulsus paradoxus is taken as the measurement derived from an arterial catheter — i.e., the difference between the highest (during expiration) and lowest (during inspiration) systolic blood pressure. 
% Pulsus paradoxus determined by a non-invasive continuous blood pressure signal (obtained using a Finapres, see Appendix C) and the PPG signal are compared with the reference measurement. 
% A respiratory cycle is determined using a strain gauge placed around the chest. 
% It is concluded that the PPG pulsus paradoxus measurement, although less accurate than the Finapres measurement shows a strong linear correlation with the reference measurement. 
% Frey et al. [36] carried out a study with the aim of evaluating the relationship between pulsus paradoxus measured intra-arterially and PPG wave changes.
%  Recordings of the PPG wave- form, arterial blood pressure and breathing cycle are taken from sixty two nonintubated patients. 
% All the analysed PPG waves and arterial blood pressure waves are in phase, with the lowest values occurring in inspiration and the highest values in expiration. 
% It is con- cluded that the PPG signal appears to be a rapid and easily performed, non-invasive method for the objective estimation of the degree of pulsus paradoxus. 
% Hartert et al. [45] studied 26 people who had severe breathing difficulties (asthma or em- physema). 
% The change in baseline in the PPG signal between expiration and inspiration was recorded. 
% The magnitude of this change was found to correlate strongly with the strength of pulsus paradoxus. 
% After treatment and clinical improvement the change in baseline between expiration and inspiration was greatly reduced, and the modulation due to respiration was no longer visible by eye.
%  These studies show that the effect of respiration can be clearly seen in the PPG of subjects with breathing difficulties such as asthma. 
% More recently it has been shown that the sensors used to measure PPG are sensitive enough to detect the pulsus paradoxus effect in the blood volume change even in healthy subjects [53].
%   \item[Pletismografo ad impedenza]
%     Il segnale IP \`e una misura dello sforzo respiratorio ed \`e usato qualche volta per visualizzare la forma d'onda della respriazione in un ambiente clinico. 
% Una corrente a basso voltaggio viene fatta passare attraverso il soggetto via elettrodi. 
% Mentre il paziente inspira ed espira, l'impedenza elettrica del petto cambia perch\'e cambia il volume d'aria nei polmoni e quindi la conduttivit\`a tra gli elettrod\`i.
%  L'aria nel torace attraversa grandi cambiamenti nel volume durante la respirazione normale ma il volume dei sangue varia anche con il ciclo cardiaco perch\'e cambia la quantit\`a di sangue nel cuore e nei vasi sanguigni. 
% Quindi l'impedenza elettrica di polmoni e cuore cambia.
%  La maggior parte dei monitor utilizzano due elettrodi standard per l'ECG posizionati sul petto. 
% L'impedanza del petto aumenta durante l'inspirazione a causa dell'aumento del volume dell'aria nel torace e del volume del sangue. 
% All'impedanza contribuiscono i musoli e il grasso ma in modo costante. Di solito le variazioni nell' impedanza toracia sono pi\`u grandi durante la respirazione che durante un ciclo cardiaco. 
% \end{description}
% 
%   \paragraph{algoritmo}
% 
% I tre segnali usati per derivare la forma d'onda respiratoria sono filtrati per eliminare ogni artefatto o componente non desiderata del segnale. 
% Ci sono molte sorgenti di rumore che causano interferenze nell'ECG. L'ECG viene filtrato attraverso un filtro passabanda in modo da eliminare ! ! !baseline wander. 
% Questo passaggio assume che la pi\`u bassa frequenza di respirazione \`e di $0.1$ respiri al minuto. 
% Anche il segnale PPG pu\`o essere disturbato. 
% Il segnale della pressione arteriosa invece non viene disturbato da rumori anche se qualche volta si vede alle basse frequenze ! ! ! baseline wander. 
% Sia il PPF che il segnale della pressione arteriosa sono stati filtrati attraverso un filtro passa banda. 
% 
% 
% riassumere 4.6.1 a 4.6.7
% 
% Algoritmi di riconoscimento dei picchi:
% 
% I picchi nel segnale respiratorio derivato sono definiti come time stamps del respiro. 
% L'algoritmo di riconoscimento dei picchi usato qui \`e un metodo basato su regole che usa un approccio di tipo gradiente. 
% Ci sono solo pochi punti per ogni ciclo di respirazione in ogni forma d'onda di respirazione derivata.
%  Dopo un detrenging, vengono cercati i cambiamenti di segno nel gradiente e abbiamo un picco o un cavo se:
% \begin{itemize}
%   \item 
%     c'\`e un cambiamento appropriato di segno(da positivo a negativo per i picchi e l'opposto per i cavi)
%   \item	
%     the previous extremum labelled is the opposite of that being detected currently
%   \item
%     il valore di punti in questione \`e sopra la media per quella sezione dei dati(o sotto la media per i cavi)
%   \item
%     il tempo trascorso da un eveto simile \`e maggiore di due secondi
% \end{itemize}
% 
% 
% Queste procedure richiedono di scegliere una finestra temporale per analizzare il segnale. 
% Se si sceglie una finestra troppo lunga, questa pu\`o essere pi\`u lunga del periodo di respirazione e quindi si sovrappongono respiri contigui in una finestra. 
% Per assicurarsi che ognuno dei respiri \`e classificato esattamente una volta la procedura \`e progettata in modo tale che il respiro trovato \`e classificato rispettivamente solo al respiro immediatamente precedente. 
% Respiri estranei dentro o fuori la finestra sono contati come falsi positivi. 
% Quindi se la finestra \`e troppo grande non si rischia di classificare due volte un respiro.
% 
% 
% 
%     
%   \paragraph{conclusioni}
%     I risultati  vengono confrontati con quelli derivati dal segnale IP. \`E possibile ottenere una sensitivit\`a del $96\%$ e una predittivit\`a positiva del $97\%$ quando si usa il segnale della pressione arteriosa. 
% Mentre nei metodi non invasivi non si ha un miglioramento significativo rispetto al metodo che usa il segnale IP.
% 
%   \paragraph{commenti}
%     I sensori scelti in questo studio, sebbene invasivi quanto uno stetoscopio elettronico, sono difficilmente reperibili e usabili in casa da personale non qualificato, inoltre sono scomodi da usare.
% 


\subsection[A Software Toolkit for Acoustic Respiratory Analysis]{ \textit{A Software Toolkit for Acoustic Respiratory Analysis} \cite{ASTFARA}}

  L'obiettivo principale di questo studio \`e la realizzazione di un software per il riconoscimento delle fasi respiratorie e per la classificazione dei suoni respiratori.
  \paragraph{input}
    Usa i suoni tracheali di cinque soggetti sani, registrati con uno stetoscopio elettronico ad una frequenza di campionamento di $22050Hz$.
  \paragraph{algoritmo}
    \begin{enumerate}
      \item 
	La fase di estrazione dei suoni polmonari rimuove la componente del segnale le cui frequenze si trovano al di fuori della banda di frequenza che va dai $100$ ai $2500Hz$, dato che i suoni respiratori tracheali di solito non cadono al di fuori di questa banda. 
	Per farlo viene usato un filtro passa banda appropriato. 
	Questo filtro usa una finestra di Blackman di dimensione pari a un dodicesimo della frequenza di campionamento del segnale di input. 
	Questo filtro \`e importante sopratutto per rimuovere i suoni ad alta frequenza che sono di origine cardiocircolatoria. 
      \item
	Una fase successiva prende il valore assoluto del segnale
      \item
	Il segnale attraversa una fase di sottocampionamento di un fattore $60$ applicando una finestra mobile di $60$ campioni che prende un campione ogni $60$. 
	Lo scopo della riduzione della frequenza di campionamento \`e quello di ridurre il carico computazionale. 
	Lo scopo della fase di trattamento del segnale \`e quello di trovare la forma generale del segnale. 
	Il sottocampionamento non degrada la forma di questa a patto che la nuova frequenza di campionamento sia almeno il doppio della frequenza di respirazione che si pu\`o evincere tipicamente dai suoni respiratori tracheali. 
	Se assumiamo che la durata tipica di una fase respiratoria \`e di un secondo allora l'envelope del segnale sar\`a di $1Hz$. 
	Questo significa che la nuova frequenza di campionamento deve essere di almeno $2Hz$. 
	In questo caso la nuova frequenza di campionamento \`e di circa $367Hz$.
      \item
	Il segnale attraversa una fase di riduzione del rumore che usa un filtro a mediana per ridurre il valore assoluto dei picchi di durata pi\`u breve. 
	Qui il segnale viene prima partizionato in segmenti contigui di $0.5s$. 
	Per ogni segmento viene calcolato il valore mediano. 
	Per ogni campione se il valore del campione supera una soglia di cinque volte il valore mediano del segmento allora il campione \`e azzerato perch\'e potrebbe essere rumore. 
      \item
	Il processo precedente viene ripetuto un'altra volta.
      \item
	Il segnale viene sottocampionato di nuovo di un fattore due.
      \item
	Il segnale passa attraverso un filtro passa basso con una finestra di Hamming di $50ms$. 
	Lo scopo di questo filtro \`e di ridurre i suoni che hanno una varianza alta e una frequenza alta.
      \item
	Applicare un filtro passa basso che ha una impulse response di lunghezza fissa e arbitraria su una segnale non stabile pu\`o portare a risultati indesiderati. 
	Poich\'e la durata delle fasi respiratorie spazia su un grande intervallo di valori, un filtro che va bene in un caso pu\`o non andare bene per un altro caso. 
	Questo problema viene affrontato usando un filtro con una impulse reponse di lunghezza che dipende dalle caratteristiche del segnale di input. 
	La lunghezza $w$ della finestra usata dal filtro dovrebbe approssimare la media della durata di una fase respiratoria nel segnale di input. 
	Per determinare la lunghezza del filtro, si parte con una fase di detrend, nella quale la media globale del segnale di input viene sottratta da ogni campione in modo tale che la media dell'output sia zero. 
      \item
	Per approssimare la media della durata di una fase respiratoria ci si basa sull'assunto che questa sia proporzionale alla ampiezza dei picchi. 
	La fase di calcolo della durata media di una fase respiratoria, approssima questa quantit\`a trovando il picco pi\`u ampio che interseca la linea di ampiezza zero. 
	Supponiamo che il segmento originato da questa intersezione abbia lunghezza $L$. 
	Allora la dimensione $w$ della finestra \`e $w=ceil(L/2)+2$. Questa scelta \`e giustificata solo empiricamente.
      \item
	Solo a questo punto possiamo applicare il filtro passa basso con una finestra di Hamming di lunghezza $w$. 
	%       This filter is applied to d[n] at the variable Hamming window. 
	% Figure 9a plots the output signal, which is significantly smoother than d[n] and takes the shape of the desired envelope of the original signal. 
	% The same the filter is applied once more to this signal in the variableHamming window2nd roundstage to further smoothen it. 
      \item	
	Usiamo informazioni nel dominio del tempo per classificare il suono di input in respiratorio o non respiratorio. 
	Questa fase usa il segnale della fase precedente e la dimensione $w$ della finestra che approssima la durata media di una fase respiratoria.
      \item
	Il segnale viene diviso in segmenti contigui di lunghezza $3w$ 
      \item
	Viene calcolata l'energia totale $E$ di un segmento sommando i valori
      \item
	L'energia $E$ viene confrontata ad una soglia. 
	La soglia \`e uguale all'energia di base  nel segmento. 
	Definiamo l'ampiezza di base come la media delle ampiezze nell'area del segnale nella quale non c'\`e respirazione e definiamo l'energia di base $E_{baseline}$ come la lunghezza del segmento moltiplicata per l'ampiezza di base. 
	Viene usato un valore fisso determinato empiricamente per l'ampiezza di base. 
	Se l'energia $E$ supera l'energia di base $E_{baseline}$ allora classifichiamo il segmento come respiro altrimenti lo classifichiamo come non respiro. 
	Viene prodotta l'etichetta $1$ se c'\`e respiro altrimenti viene prodotta l'etichetta $0$.
      \item
	La fase precedente produce una sequenza di etichette. 
	Etichette uguali e contigue vengono raggruppate.
      \item
	Un respiro \`e un gruppo di etichette $1$ seguito da un gruppo di etichette $0$.
    \end{enumerate}

    Il sistema ha una accuratezza del $97\%$




c pulsations in arterial blood volume.
% %  These periodic pulsations give rise to the blood pressure waveform. 
% % Several studies have attempted to determine the correlation between the respiration induced effects in the blood pressure and PPG signals, the aim of these studies being to investigate the possibility of non-invasive analysis of pulsus paradoxus using PPG. 
% % Steele et al. [97] report on different methods of continuous non-invasive monitoring of pulsus paradoxus.
% %  The reference measurement of pulsus paradoxus is taken as the measurement derived from an arterial catheter — i.e., the difference between the highest (during expiration) and lowest (during inspiration) systolic blood pressure. 
% % Pulsus paradoxus determined by a non-invasive continuous blood pressure signal (obtained using a Finapres, see Appendix C) and the PPG signal are compared with the reference measurement. 
% % A respiratory cycle is determined using a strain gauge placed around the chest. 
% % It is concluded that the PPG pulsus paradoxus measurement, although less accurate than the Finapres measurement shows a strong linear correlation with the reference measurement. 
% % Frey et al. [36] carried out a study with the aim of evaluating the relationship between pulsus paradoxus measured intra-arterially and PPG wave changes.
% %  Recordings of the PPG wave- form, arterial blood pressure and breathing cycle are taken from sixty two nonintubated patients. 
% % All the analysed PPG waves and arterial blood pressure waves are in phase, with the lowest values occurring in inspiration and the highest values in expiration. 
% % It is con- cluded that the PPG signal appears to be a rapid and easily performed, non-invasive method for the objective estimation of the degree of pulsus paradoxus. 
% % Hartert et al. [45] studied 26 people who had severe breathing difficulties (asthma or em- physema). 
% % The change in baseline in the PPG signal between expiration and inspiration was recorded. 
% % The magnitude of this change was found to correlate strongly with the strength of pulsus paradoxus. 
% % After treatment and clinical improvement the change in baseline between expiration and inspiration was greatly reduced, and the modulation due to respiration was no longer visible by eye.
% %  These studies show that the effect of respiration can be clearly seen in the PPG of subjects with breathing difficulties such as asthma. 
% % More recently it has been shown that the sensors used to measure PPG are sensitive enough to detect the pulsus paradoxus effect in the blood volume change even in healthy subjects [53].
% %   \item[Pletismografo ad impedenza]
% %     Il segnale IP \`e una misura dello sforzo respiratorio ed \`e usato qualche volta per visualizzare la forma d'onda della respriazione in un ambiente clinico. 
% % Una corrente a basso voltaggio viene fatta passare attraverso il soggetto via elettrodi. 
% % Mentre il paziente inspira ed espira, l'impedenza elettrica del petto cambia perch\'e cambia il volume d'aria nei polmoni e quindi la conduttivit\`a tra gli elettrod\`i.
% %  L'aria nel torace attraversa grandi cambiamenti nel volume durante la respirazione normale ma il volume dei sangue varia anche con il ciclo cardiaco perch\'e cambia la quantit\`a di sangue nel cuore e nei vasi sanguigni. 
% % Quindi l'impedenza elettrica di polmoni e cuore cambia.
% %  La maggior parte dei monitor utilizzano due elettrodi standard per l'ECG posizionati sul petto. 
% % L'impedanza del petto aumenta durante l'inspirazione a causa dell'aumento del volume dell'aria nel torace e del volume del sangue. 
% % All'impedanza contribuiscono i musoli e il grasso ma in modo costante. Di solito le variazioni nell' impedanza toracia sono pi\`u grandi durante la respirazione che durante un ciclo cardiaco. 
% % \end{description}
% % 
% %   \paragraph{algoritmo}
% % 
% % I tre segnali usati per derivare la forma d'onda respiratoria sono filtrati per eliminare ogni artefatto o componente non desiderata del segnale. 
% % Ci sono molte sorgenti di rumore che causano interferenze nell'ECG. L'ECG viene filtrato attraverso un filtro passabanda in modo da eliminare ! ! !baseline wander. 
% % Questo passaggio assume che la pi\`u bassa frequenza di respirazione \`e di $0.1$ respiri al minuto. 
% % Anche il segnale PPG pu\`o essere disturbato. 
% % Il segnale della pressione arteriosa invece non viene disturbato da rumori anche se qualche volta si vede alle basse frequenze ! ! ! baseline wander. 
% % Sia il PPF che il segnale della pressione arteriosa sono stati filtrati attraverso un filtro passa banda. 
% % 
% % 
% % riassumere 4.6.1 a 4.6.7
% % 
% % Algoritmi di riconoscimento dei picchi:
% % 
% % I picchi nel segnale respiratorio derivato sono definiti come time stamps del respiro. 
% % L'algoritmo di riconoscimento dei picchi usato qui \`e un metodo basato su regole che usa un approccio di tipo gradiente. 
% % Ci sono solo pochi punti per ogni ciclo di respirazione in ogni forma d'onda di respirazione derivata.
% %  Dopo un detrenging, vengono cercati i cambiamenti di segno nel gradiente e abbiamo un picco o un cavo se:
% % \begin{itemize}
% %   \item 
% %     c'\`e un cambiamento appropriato di segno(da positivo a negativo per i picchi e l'opposto per i cavi)
% %   \item	
% %     the previous extremum labelled is the opposite of that being detected currently
% %   \item
% %     il valore di punti in questione \`e sopra la media per quella sezione dei dati(o sotto la media per i cavi)
% %   \item
% %     il tempo trascorso da un eveto simile \`e maggiore di due secondi
% % \end{itemize}
% % 
% % 
% % Queste procedure richiedono di scegliere una finestra temporale per analizzare il segnale. 
% % Se si sceglie una finestra troppo lunga, questa pu\`o essere pi\`u lunga del periodo di respirazione e quindi si sovrappongono respiri contigui in una finestra. 
% % Per assicurarsi che ognuno dei respiri \`e classificato esattamente una volta la procedura \`e progettata in modo tale che il respiro trovato \`e classificato rispettivamente solo al respiro immediatamente precedente. 
% % Respiri estranei dentro o fuori la finestra sono contati come falsi positivi. 
% % Quindi se la finestra \`e troppo grande non si rischia di classificare due volte un respiro.
% % 
% % 
% % 
% %     
% %   \paragraph{conclusioni}
% %     I risultati  vengono confrontati con quelli derivati dal segnale IP. \`E possibile ottenere una sensitivit\`a del $96\%$ e una predittivit\`a positiva del $97\%$ quando si usa il segnale della pressione arteriosa. 
% % Mentre nei metodi non invasivi non si ha un miglioramento significativo rispetto al metodo che usa il segnale IP.
% % 
% %   \paragraph{commenti}
% %     I sensori scelti in questo studio, sebbene invasivi quanto uno stetoscopio elettronico, sono difficilmente reperibili e usabili in casa da personale non qualificato, inoltre sono scomodi da usare.
% % 
% 
% 
% \subsection[A Software Toolkit for Acoustic Respiratory Analysis]{ \textit{A Software Toolkit for Acoustic Respiratory Analysis} \cite{ASTFARA}}
% 
%   L'obiettivo principale di questo studio \`e la realizzazione di un software per il riconoscimento delle fasi respiratorie e per la classificazione dei suoni respiratori.
%   \paragraph{input}
%     Usa i suoni tracheali di cinque soggetti sani, registrati con uno stetoscopio elettronico ad una frequenza di campionamento di $22050Hz$.
%   \paragraph{algoritmo}
%     \begin{enumerate}
%       \item 
% 	La fase di estrazione dei suoni polmonari rimuove la componente del segnale le cui frequenze si trovano al di fuori della banda di frequenza che va dai $100$ ai $2500Hz$, dato che i suoni respiratori tracheali di solito non cadono al di fuori di questa banda. 
% 	Per farlo viene usato un filtro passa banda appropriato. 
% 	Questo filtro usa una finestra di Blackman di dimensione pari a un dodicesimo della frequenza di campionamento del segnale di input. 
% 	Questo filtro \`e importante sopratutto per rimuovere i suoni ad alta frequenza che sono di origine cardiocircolatoria. 
%       \item
% 	Una fase successiva prende il valore assoluto del segnale
%       \item
% 	Il segnale attraversa una fase di sottocampionamento di un fattore $60$ applicando una finestra mobile di $60$ campioni che prende un campione ogni $60$. 
% 	Lo scopo della riduzione della frequenza di campionamento \`e quello di ridurre il carico computazionale. 
% 	Lo scopo della fase di trattamento del segnale \`e quello di trovare la forma generale del segnale. 
% 	Il sottocampionamento non degrada la forma di questa a patto che la nuova frequenza di campionamento sia almeno il doppio della frequenza di respirazione che si pu\`o evincere tipicamente dai suoni respiratori tracheali. 
% 	Se assumiamo che la durata tipica di una fase respiratoria \`e di un secondo allora l'envelope del segnale sar\`a di $1Hz$. 
% 	Questo significa che la nuova frequenza di campionamento deve essere di almeno $2Hz$. 
% 	In questo caso la nuova frequenza di campionamento \`e di circa $367Hz$.
%       \item
% 	Il segnale attraversa una fase di riduzione del rumore che usa un filtro a mediana per ridurre il valore assoluto dei picchi di durata pi\`u breve. 
% 	Qui il segnale viene prima partizionato in segmenti contigui di $0.5s$. 
% 	Per ogni segmento viene calcolato il valore mediano. 
% 	Per ogni campione se il valore del campione supera una soglia di cinque volte il valore mediano del segmento allora il campione \`e azzerato perch\'e potrebbe essere rumore. 
%       \item
% 	Il processo precedente viene ripetuto un'altra volta.
%       \item
% 	Il segnale viene sottocampionato di nuovo di un fattore due.
%       \item
% 	Il segnale passa attraverso un filtro passa basso con una finestra di Hamming di $50ms$. 
% 	Lo scopo di questo filtro \`e di ridurre i suoni che hanno una varianza alta e una frequenza alta.
%       \item
% 	Applicare un filtro passa basso che ha una impulse response di lunghezza fissa e arbitraria su una segnale non stabile pu\`o portare a risultati indesiderati. 
% 	Poich\'e la durata delle fasi respiratorie spazia su un grande intervallo di valori, un filtro che va bene in un caso pu\`o non andare bene per un altro caso. 
% 	Questo problema viene affrontato usando un filtro con una impulse reponse di lunghezza che dipende dalle caratteristiche del segnale di input. 
% 	La lunghezza $w$ della finestra usata dal filtro dovrebbe approssimare la media della durata di una fase respiratoria nel segnale di input. 
% 	Per determinare la lunghezza del filtro, si parte con una fase di detrend, nella quale la media globale del segnale di input viene sottratta da ogni campione in modo tale che la media dell'output sia zero. 
%       \item
% 	Per approssimare la media della durata di una fase respiratoria ci si basa sull'assunto che questa sia proporzionale alla ampiezza dei picchi. 
% 	La fase di calcolo della durata media di una fase respiratoria, approssima questa quantit\`a trovando il picco pi\`u ampio che interseca la linea di ampiezza zero. 
% 	Supponiamo che il segmento originato da questa intersezione abbia lunghezza $L$. 
% 	Allora la dimensione $w$ della finestra \`e $w=ceil(L/2)+2$. Questa scelta \`e giustificata solo empiricamente.
%       \item
% 	Solo a questo punto possiamo applicare il filtro passa basso con una finestra di Hamming di lunghezza $w$. 
% 	%       This filter is applied to d[n] at the variable Hamming window. 
% 	% Figure 9a plots the output signal, which is significantly smoother than d[n] and takes the shape of the desired envelope of the original signal. 
% 	% The same the filter is applied once more to this signal in the variableHamming window2nd roundstage to further smoothen it. 
%       \item	
% 	Usiamo informazioni nel dominio del tempo per classificare il suono di input in respiratorio o non respiratorio. 
% 	Questa fase usa il segnale della fase precedente e la dimensione $w$ della finestra che approssima la durata media di una fase respiratoria.
%       \item
% 	Il segnale viene diviso in segmenti contigui di lunghezza $3w$ 
%       \item
% 	Viene calcolata l'energia totale $E$ di un segmento sommando i valori
%       \item
% 	L'energia $E$ viene confrontata ad una soglia. 
% 	La soglia \`e uguale all'energia di base  nel segmento. 
% 	Definiamo l'ampiezza di base come la media delle ampiezze nell'area del segnale nella quale non c'\`e respirazione e definiamo l'energia di base $E_{baseline}$ come la lunghezza del segmento moltiplicata per l'ampiezza di base. 
% 	Viene usato un valore fisso determinato empiricamente per l'ampiezza di base. 
% 	Se l'energia $E$ supera l'energia di base $E_{baseline}$ allora classifichiamo il segmento come respiro altrimenti lo classifichiamo come non respiro. 
% 	Viene prodotta l'etichetta $1$ se c'\`e respiro altrimenti viene prodotta l'etichetta $0$.
%       \item
% 	La fase precedente produce una sequenza di etichette. 
% 	Etichette uguali e contigue vengono raggruppate.
%       \item
% 	Un respiro \`e un gruppo di etichette $1$ seguito da un gruppo di etichette $0$.
%     \end{enumerate}
% 
%     Il sistema ha una accuratezza del $97\%$
% 
% 
% 
% 
\end{frame}