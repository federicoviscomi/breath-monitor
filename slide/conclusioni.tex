
\section{Sviluppi futuri}
\begin{frame}
\frametitle{Sviluppi futuri}

\begin{itemize}
  \item 
    Classificare i suoni respiratori (normali, anormali, soffi, crackles, ...).
  \item
    Riconoscere gli schemi respiratori (normale, Cheyne Stokes, agonico, Kussmaul, ...).
  \item
    Implementare un'interfaccia con uno stetoscopio elettronico.
  \item
    Implementare un algoritmo di stima del flusso.
  \item
    Portare il software su un dispositivo mobile. 
  \item
    Creare un database di registrazioni di suoni respiratori.    
  \item
    Creare un database di casi di test.
  \item
    Implementare dei meccanismi di tolleranza al rumore esterno. Ad esempio attraverso la creazione di un modello acustico approssimato del torace.
\end{itemize}

% 	
% 	Ad esempio usare due microfoni: uno che registra il rumore ambientale e uno che registra i suoni respiratori e usare il modello acustico per estrarre il rumore ambientale dai suoni respiratori.
%       \item
% 	Fare una analisi approfondita dello stato dell'arte della separazione dei suoni cardiovascolari dai suoni respiratori, partendo ad esempio dagli articoli \cite{separazione1, separazione2, separazione3, separazione4, separazione5, separazione6}.
%     \end{itemize}
\end{frame}









\section{Conclusioni}
\begin{frame}
\frametitle{Conclusioni}
% \framesubtitle{Commenti}


\begin{description}
  \item[Obiettivi]
    Progettare, prototipare e valutare un software di riconoscimento delle apnee notturne attraverso uno stetoscopio elettronico applicato sul petto o sulla trachea di un soggetto.
  \item[Risultati] 
    I risultati raggiunti sono incoraggianti e
    fanno da un punto di partenza verso un sistema usabile in uno scenario reale.    
\end{description}

% Gli obbiettivi di questa tesi sono di progettare, prototipare e valutare un software di riconoscimento delle apnee notturne attraverso uno stetoscopio elettronico applicato sul petto o sulla trachea di un soggetto.
%  Ricapitolare quello che e' stato fatto. ricordare qual'era l'obbiettivo, dire che il sistema e' stato implementato e testato con risultati incoraggianti




\end{frame}