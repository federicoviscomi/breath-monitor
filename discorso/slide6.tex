\section{Slide 6 Motivazioni}

I motivi che suggeriscono l'utilit\`a del sistema sutidato in questa tesi sono:
\begin{description}
 \item[Diagnosi] 
    Il sistema \`e utile come strumento di diagnosi della sindrome da apena del sonno. E\`e uno strumento meno invasivo e costoso rispetto alla sonnografia con spirometro o pulsiossimentro.
 \item[Terapia d'emergenza]
  Il sistema pu\`o svegliare il soggetto se questo si trova in una fase di apena respiratoria troppo lunga. 
\end{description}

Rimane un problema aperto di come scegliere la soglia di allarme e se convenga o meno svegliare il soggetto.

 La soglia oltre la quale una apnea \`e considerata pericolosa \`e configurabile. 
 Il valore soglia deve essere stabilito da personale medico qualificato. 
 Si pu\`o intuire che una soglia troppo bassa potrebbe degradare la qualit\`a del sonno del soggetto in un modo patogenico o quantomeno in un modo tale da rendere inutile il monitoraggio. 
 Al contrario una soglia troppo alta espone il soggetto ad un rischio troppo elevato. 
 Ci si pu\`o aspettare che la soglia di allarme non sia oggettiva ma debba essere personalizzata e che vari con l'et\`a e alcuni parametri fisiologici del soggetto.


