
\section{Slide 5 Sindrome da apnea del sonno}
La \emph{sindrome da apnea del sonno} \`e un disordine del sonno molto diffuso.
% caratterizzato da ripetute apnee durante il sonno. 
% La \emph{sindrome da apnea del sonno} \`e un disordine del sonno caratterizzato da ripetute apnee o ipopnee durante il sonno. 
% Una \emph{ipopnea} \`e un evento caratterizzato da respirazione insufficiente, pi\`u precisamente una ipopnea si verifica quando il flusso d'aria si riduce di almeno il $30\%$ per almeno dieci secondi e la desaturazione di ossigeno nel sangue \`e di almeno il $4\%$ . 

% La sindrome da apnea del sonno \`e una patologia molto diffusa. 
Il sonno di un soggetto affetto da tale patologia \`e disturbato da frequenti apnee e da episodi di respirazione insufficiente. 


Le apnee del sonno si verificano sia nei bambini che negli adulti. 
I soggetti affetti da apnea del sonno possono manifestare i seguenti sintomi: eccessiva sonnolenza durante il giorno, spossatezza, tempi di reazione lenti, problemi alla vista, indebolimento delle funzioni del fegato e altro.
Forme medie e gravi di sindrome da apnea del sonno sono un fattore di rischio per, e una concausa di: pressione alta, malattie cardiache, diabete, depressione. 

% L'\emph{indice di apnea-ipopnea(apnea-hypopnea index AHI)} \`e definito come il numero di eventi di apnea e di ipopnea in rapporto alla durata del sonno. 
% L'AHI \`e un indicatore della gravit\`a della sindrome di apnea del sonno e i suoi valori sono categorizzati tipicamente in: leggera da 5 a 15 episodi all'ora, moderata da 15 a 30 episodi all'ora o severa oltre i 30 episodi all'ora. 

La sindrome da apnea del sonno di solito viene diagnosticata in centri specializzati nei quali si monitorano alcuni parametri biomedici di un soggetto mentre dorme.
Si stima che pi\`u della met\`a dei soggetti affetti da tale patologia non ne siano al corrente. 
Questo a causa dei meccanismi di diagnosi pi\`u diffusi che sono costosi o scomodi e richiedono al paziente di trascorrere la notte in un centro specializzato.



