\section{Slide 7 Conviene svegliare il soggetto?}
L'articolo citato nella nota lascia intuire che conviene svegliare il soggetto.
Prende in esame i polisonnogrammi e i certificati di morte di alcune persone che sono decedute a causa di una malattia cardiaca improvvisa. 
L'articolo divide il campione in due gruppi: le persone nel primo gruppo soffrivano di sindrome da apnea del sonno mentre nel secondo gruppo no.
La percentuale dei casi di morte durante il sonno nel primo gruppo \`e del $46\%$ mentre la percentuale dei casi di morte durante il sonno nel secondo gruppo \`e molto minore $21\%$ e nella popolazione generale \`e del $16\%$.
% Lo studio conclude che la gravit\`a della sindrome da apnea del sonno \`e direttamente proporzionale al rischio di morte improvvisa per malattie cardiache durante il sonno. 
% Episodi acuti di apnea o ipopnea possono indurre: ipossiemia, aumento degli impulsi nel sistema nervoso simpatico, aumento brusco nella pressione sanguigna, aumento dello stress delle pareti cardiache, aritmie cardiache, ipercoagulabilit\`a, stress ossidativo vascolare, infiammazioni sistemiche e altro. 
% Questo potrebbe spiegare i dati osservati dallo studio. 
Resta un problema aperto quello di stabilire se nel primo gruppo di persone, la morte sia immediatamente preceduta da un evento di apnea grave.
