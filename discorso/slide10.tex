\section{Slide 10 Breath Analysis of Respiratory Flow using Tracheal Sounds}

L'articolo citato studia due parametri biomedici di alcuni soggetti sani non fumatori i quali non hanno mai avuto gravi malattie respiratorie.
Nella parte in alto del grafico vediamo il flusso d'aria misurato con uno spirometro con pneumotacografo.
Il flusso \`e riportato in litri al secondo.
Nella parte in basso del grafico vediamo la media dell'energia del segnale presa in finestre rettangolari da $50ms$ con sovrapposizione del $75\%$ prima tra alcune specifiche bande di frequenza e poi tra i vari soggetti.
Questa media viene normalizzata rispetto al valore massimo.

%        \begin{enumerate}
%   	\item
%  	  Nelle fasi seguenti l'algoritmo considera solo le porzioni del suono registrate quando il segnale del flusso era al di sotto del $20\%$ del flusso medio o al di sopra del $20\%$ di esso.
%  	  Perch\'e in queste condizioni il suono tracheale si pu\`o considerare stazionario.
%  	\item
%  	  Lo spettro di potenza dei suoni della trachea \`e stato calcolato in una finestra di $50ms$ ($512$ campioni) con il $75\%$ di sovrapposizione tra finestre successive. 
% 	  \`e stata calcolata la media dell'energia dei suoni tracheali entro sei predefinite bande di frequenza: da $70$ a $300Hz$, da $300$ a $450Hz$, da $450$ a $600Hz$, da $600$ a $800Hz$, da $800$ a $1000Hz$ e da $1000$ a $1200Hz$. 
% 	\item
% 	  Dato che le intensit\`a dei suoni respiratori variano da soggetto a soggetto, per ogni soggetto i valori calcolati in precedenza sono stati normalizzati rispetto al valore massimo.
% 	\item
% 	  Si \`e poi calcolata la media dei valori normalizzati tra soggetti diversi per ogni fase respiratoria, inoltre sono state calcolate la media e l'errore standard per diversi tassi di flusso e intervalli di frequenza.
%       \end{enumerate} 
%       mentre il secondo flusso di esecuzione \`e:
%       \begin{enumerate}
% 	\item
% 	  Il segnale dei suoni della trachea sono stati filtrati attraverso un filtro passa alto nelle stesse frequenze menzionate in precedenza. 
% 	\item
% 	  Il segnale filtrato \`e stato in seguito segmentato in finestre di dimensione $50ms$ ($512$ campioni) con il $75\%$ di sovrapposizione tra finestre successive usando una finestra di Hanning. 
% 	\item
% 	  Si calcola il logaritmo della varianza dei segmenti precedenti.
% 	\item
% 	  In ciascuna finestra il valore precedente viene normalizzato rispetto al valore massimo per ridurre le interferenze dei suoni del cuore
% 	\item	
% 	  In seguito viene calcolata la media all'interno delle diverse bande di frequenza e diversi valori del flusso d'aria.
%       \end{enumerate}
%     

% la media dell'energia normalizzata del suono
      Dal grafico si pu\`o vedere che il secondo parametro \`e proporzionale al valore assoluto del flusso. 

%        Inoltre anche la varianza logaritmica normalizzata e la media di potenza normalizzata seguono i cambiamenti nel valore assoluto del flusso. 
%        Nello spettrogramma, nella varianza logaritmica normalizzata e nella media normalizzata della potenza sono evidenti le transizioni di fase respiratoria. 
%        Quando il flusso era medio o alto, si ha una maggiore differenza di media dell'energia normalizzata tra la fase inspiratoria ed espiratoria nella banda di frequenze dai $300$ ai $450Hz$. 
%        Inoltre questa banda di frequenza ottiene la seconda maggior differenza di media dell'energia normalizzata tra la fase inspiratoria e la fase esipiratoria quando il flusso \`e basso o molto alto. 
%        Quindi questo intervallo di frequenza \`e stato scelto come ottimale per esaminare i cambiamenti nella media della potenza rispettivamente alle fasi respiratorie.
