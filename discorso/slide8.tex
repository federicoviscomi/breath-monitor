\section{Slide 8 Schema di funzionamento}


% Il software pu\`o rientrare in due categorie:
% \begin{itemize}
%   \item
%     La prima categoria si chiama \emph{auscultazione assistita dal calcolatore}, alla quale afferiscono sistemi di supporto alle decisioni cliniche, progettati per aiutare il medico nella diagnosi attraverso i suoni del corpo. 
%     In questo caso gli utenti del sistema fanno parte del personale medico-sanitario di una struttura clinica.
%   \item
%     La seconda categoria \`e il monitoraggio mobile di segnali biomedici. 
%     In questo caso gli utenti del sistema possono essere soggetti privi di conoscenze mediche o infermieristiche.
% \end{itemize}
% 
%    \begin{table}
%   \centering
%   \begin{tabular}{p{0.17\textwidth} p{0.77\textwidth}}
%   \\
%   \hline
%       Caso d'uso 
%     & 
%       Attiva monitoraggio
%   \\\hline\\
%       Attori
%     &
%       C'\`e un solo attore attivo: un soggetto che intende monitorare il proprio respiro oppure un membro del personale medico sanitario che intende monitorare il respiro di un paziente. Il soggetto del quale si misura il respiro \`e un attore passivo in questo caso d'uso perch\'e in generale esso pu\`o evitare di interagire in modo attivo col sistema stesso. 
%   \\
%       Precondizioni
%     &
%       \begin{itemize}
% 	\item 
% 	  Lo stetoscopio elettronico \`e stato installato correttamente sul torace del soggetto o sulla trachea del soggetto.
% 	\item
% 	  I dispositivi di interfaccia tra lo stetoscopio e il sistema sono configurate correttamente.
%       \end{itemize}
%   \\
%       Sequenza principale degli eventi
%     &
%       \begin{enumerate}
% 	\item 
% 	  L'attore attiva il sistema di monitoraggio attraverso l'interfaccia utente. 
% 	  In questo passo l'attore specifica se intende memorizzare o meno i dati e in che forma.
% 	  L'attore pu\`o specificare la soglia di allarme per l'apnea.
% 	\item
% 	  Il sistema stabilisce una connessione con lo stetoscopio elettronico. 
% 	  Se questo passo fallisce allora comincia la sequenza alternativa degli eventi.
% 	\item	
% 	  Il sistema passa in fase di monitoraggio.
% 	\item
% 	  Se la frequenza di respirazione scende al disotto di una certa soglia critica, il sistema lo segnala attraverso una parte dedicata dell'interfaccia utente, ad esempio un segnale acustico.
%       \end{enumerate}      
%   \\
%       Sequenza alternativa degli eventi
%     &
%       \begin{enumerate}
% 	\item 
% 	  Il sistema segnala un errore appropriato attraverso l'interfaccia utente.
%       \end{enumerate}      
%   \\
%       Postcondizioni
%     &
%       Il sistema di monitoraggio \`e attivo. 
%   \\\\
%   \hline
%   \end{tabular}
%    \caption{Caso d'uso: attiva monitoraggio}
%    \label{casoDUsoMonitoraggio}
%    \end{table}


Il caso d'uso principale ha la seguente sequenza degli eventi principale:
\begin{description}
  \item[Acquisizione dati dallo stetoscopio elettronico]
    Il sistema acquisisce un blocco di segnale dallo stetoscopio elettronico ad esempio un secondo di segnale.
%  In generale i sistemi di monitoraggio continuo di parametri fisiologici hanno bisogno di interfacciarsi con i sensori di interesse. 
% Questo sistema non \`e una eccezione. 
% Riteniamo che una interfaccia minima con lo stetoscopio elettronico sia quella illustrata nella figura . Le operazioni illustrate sono:
% \begin{description}
%   \item[$connect$]
%     Inizializza la connessione con il sensore ed allocare eventuali risorse necessarie.
%   \item[$disconnect$]
%     Terminare la connessione con il sensore e deallocare eventuali risorse.
%   \item[$read$]
%     Legge $length$ campioni di input e li memorizza nel buffer a partire dall'offset specificato.
%   \item[$skip$]
%     Tralascia un certo numero di campioni. Puo' essere utile per recuperare in parte un eventuale ritardo globale.
% \end{description}
% 
% 
% Il sistema deve includere almeno una componente che implementa l'interfaccia tra il dispositivo di monitoraggio e il sensore.
% Alcune opzioni sono illustrate in figura
% 
Ci sono vari modi di implementare una connessione tra lo stetoscopio e il dispositivo di monitoraggio. 
% Una scelta che dobbiamo fare riguarda il mezzo di trasmissione del segnale:
% \begin{description}
%   \item[via cavo]
%     In questa modalit\`a spesso \`e necessaria la presenza di personale medico specializzato. Tra gli svantaggi di questa impostazione notiamo: ridotta mobilit\`a del sistema, probabile necessit\`a di alimentazione elettrica dalla rete, rischio di gestione inadeguata di contatti nei cavi.
%   \item[senza cavo]
%     Una connessione senza fili presenta alcuni vantaggi ad esempio permette al paziente di muoversi pi\`u liberamente e i costi di installazione si possono ridurre.
% \end{description}
% 
% Riteniamo che per questo sistema sia pi\`u adeguata una connessione senza fili. 
% 
% Un'altra scelta riguarda il protocollo di comunicazione, il quale per\`o potrebbe dipendere anche dalla scelta del mezzo di trasmissione.
% 
% 
% Siamo in presenza di un problema la cui modellazione porta naturalmente a pensare ad un pattern architetturale di tipo client server. 
% Un server \`e un dispositivo fisico o virtuale che possiede una risorsa da condividere, in questo caso il server \`e lo stetoscopio e la risorsa da condividere \`e il suono che esso registra. Un cliente \`e un dispositivo fisico o virtuale che richiede una certa risorsa ad un server, in questo caso il client \`e il dispositivo di analisi del suono.
% 
% \paragraph{bluethoot}
% In base alle considerazioni fatte da , la teconologia wireless bluethoot si adatta bene al nostro sistema. 
% 
Ad esempio la connessione pu\`o essere wireless bluethoot.
Il bluethoot permette di stabilire semplici connessioni ad hoc tra dispositivi che hanno a disposizione poca energia elettrica e che sono posti a piccola distanza tra di loro.
% dove per piccola distanza approssimativamente intendiamo che i dispositivi si trovano nella stessa stanza o anche entro certi limiti in una stessa struttura ospedaliera. 
% I valori precisi di consumi, distanze e velocit\`a di trasmissione variano da dispositivo a dispositivo. 
% Lo stetoscopio elettronico invia continuamente dati al sistema di riconoscimento attraverso il canale bluethoot. 
% I dati inviati dipendono dal particolare stetoscopio ma possiamo aspettarci che questi siano sotto forma di pacchetti di un segnale audio digitale. 
% 
% \paragraph{velocit\`a dell'intefaccia}
% 
% Supponiamo che il sistema abbia una velocit\`a media di esecuzione di $v_{s}$ campioni di segnale al secondo, e supponiamo che $v$ sia maggiore della frequenza di campionamento e che quindi il sistema riesca ad analizzare un segnale di un secondo in un tempo minore di un secondo. Questa velocit\`a si intende calcolata senza contare il tempo di trasmissione dallo stetoscopio al sistema.
% 
%  Supponiamo che la velocit\`a di trasmissione dell'interfaccia tra sistema e stetoscopio sia $v_{i}$ bit al secondo e siano $f_{c}$ la frequenza di campionamento del segnale e $size$ la dimensione in bit di un campione di segnale. Allora deve valere
%  \[
%    \displaystyle  \frac{f_{c}}{v_{s}} + \displaystyle \frac{f_{c}}{\left\lfloor\frac{v_{i}}{size}\right\rfloor} \leq 1s
%  \]
%  affinch\'e il sistema non accumuli ritardo.

  \item[Estrazione dei suoni respiratori]
    In questa fase il sistema cerca di isolare i suoni respiratori da qualli cardiaci e dai rumori esterni. 
    Ne parlo pi\`u in dettaglio in una slide successiva.
  \item[Riconoscimento delle apnee]
    In questa fase il sistema cerca di riconoscere le pause respiratorie e tiene traccia della durata totale delle ultime pause consecutive.
  \item[Caso di allarme?]
    Se la durata totale delle ultime pause consecutive \`e maggiore della soglia di rischio allora il sistema suona una allarme e si mette in attesa che questa venga disattivata attraverso l'interfaccia utente.
    Altrimenti il ciclo si ripete.
\end{description}

In un certo senso quindi l'output \`e la presenza o l'assenza di una pausa respiratoria nel blocco di segnale acquisito al punto inziale.
% 
% \subsection{Interfaccia utente}
% 
% Sono state implementate due interfacce minimali: 
% \begin{description}
%   \item[console]
%     Un interprete di comandi da console che ha questo aspetto:
%     \begin{verbatim}
% Breath Monitor 
% by Federico Viscomi
% type help for a command list
% 
% $ help
% 
% exit                	exit application             
% help                	print this help               
% list                	list working directory content
% monitor -f filename 	start breath recognition on given file name
% stop                	stop current monitoring if any
% $ 
%     \end{verbatim}
% 
%   \item[GUI]
%     Una interfaccia grafica basata su Swing che offre le stesse funzioni di quella grafica e in pi\`u usa la libreria open source JMathPlot  per disegnare il grafico nel dominio del tempo del segnale. 
%     Questo grafico \`e utile solo nella versione iniziale del sistema per motivi di debug. 
%     Mentre in una versione successiva del sistema si pu\`o rimpiazzare questo grafico con quello del flusso d'aria. 
%     La finestra principale \`e illustrata in figura 
% \end{description}
% 



% In un sistema real time di elaborazione di segnali digitali, il segnale in input pu\`o essere virtualmente illimitato nel tempo. 
% In realt\`a dei valori che massimizzano la durata del segnale si potrebbero trovare ma sono abbastanza grandi da costringerci ad usare una particolare definizione di scadenze temporali. 
% Il ritardo nell'elaborazione deve essere limitato anche se il processo continua per un tempo illimitato. 
% Quindi consideriamo la media del tempo di elaborazione del segnale per campione di segnale in un intervallo di tempo abbastanza piccolo rispetto ai vincoli real time, ad esempio un secondo. 
% Questa media non deve essere maggiore del periodo di campionamento. 
% Questo criterio vale sia che il segnale venga esaminato in blocchi sia che il segnale venga esaminato campione per campione
% In altre parole un sistema real time di elaborazione di un segnale virtualmente illimitato deve avere un tempo di esecuzione per secondo di segnale, minore di un secondo.


% La velocit\`a di esecuzione un algoritmo \`e una grandezza data dal rapporto tra la dimensione dell'input e il tempo di esecuzione su di esso.
% % In questo caso la dimensione dell'input \`e la durata del segnale e non ci interessa la velocit\`a calcolata su tutto il segnale di input ma quella calcolata ad intervalli regolari di dimensione piccola rispetto ai vincoli real time.
% Quindi si pu\`o definire la velocit\`a $v$ del processo come la quantit\`a di campioni che ci sono in un secondo di segnale, fratto il tempo impiegato per l'elaborazione di un secondo di segnale. 
% In un secondo di segnale ci sono un numero di campioni pari alla frequenza di campionamento del segnale $f_{c} Hz$.
% Il sistema pu\`o calcolare tale velocit\`a ogni secondo e quindi produrre una sequenza di velocit\`a $v_{1}, v_{2}, \cdots $. 
% Una condizione necessaria affinch\'e il sistema si trovi sempre (ogni secondo) in uno stato valido \`e la seguente:
% \[
%   \forall n.\; v_{n} \geq f_{c}
% \]
% 
% 
% Il sistema in fase di monitoraggio deve segnalare la presenza di apnee troppo lunghe. 
% In particolare consideriamo troppo lunga una apnea di $30$ secondi. 
% Partiamo dall'assunto che le sole scadenze siano quelle relative all'evento apnea troppo lunga e che non rispettare una scadenza significa non dare l'allarme in tempo prima che il soggetto rischi gravi problemi cardiorespiratori. 
% Allora ci troviamo in presenza di un sistema real time hard. 
% Tuttavia i vincoli di tempo per le scadenze sono blandi relativamente ai tempi di esecuzione che si possono prevedere su un calcolatore moderno.
% 


