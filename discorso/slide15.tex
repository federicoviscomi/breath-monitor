\section{Slide 15 Creare i casi di test}

Per la scelta dei casi di test usiamo un approccio di tipo black-box e quindi esaminiamo da prima lo spazio dell'input e poi i possibili scenari di uso. 
Lo spazio dell'input \`e un sottoinsieme del tipo dell'input nel quale rientrano tutti i segnali audio che possono essere ascoltati da uno stetoscopio elettronico posizionato sul petto di un soggetto.
I test fatti sono di tipo \emph{oracolo} nel senso che l'output dell'algoritmo in ogni caso di test viene confrontato con il risultato che l'algoritmo dovrebbe fornire. 
Ad esempio alcuni casi di test possono avere come input:
\begin{itemize}
  \item
    Un file audio abbastanza lungo da simulare un monitoraggio del sonno reale. 
    Lo scopo di un caso d'uso con questo input \`e la valutazione della velocit\`a a lungo termine dell'algoritmo.
  \item
    Dei suoni respiratori sovrapposti a rumore di vari tipo ed intensit\`a. 
    Lo scopo di un caso d'uso con questo input \`e la valutazione della tolleranza al rumore.
    Per creare dei casi che valutano la resistenza al rumore si pu\`o procedere nel modo seguente:
    \begin{enumerate}
      \item 
	Scegliere una file contenente una sorgente di rumore e scegliere una intensit\`a della sorgente di rumore.
      \item
	Filtrare il file di rumore in base ad un certo modello acustico del corpo. 
	Cio\`e cercare di prevedere cosa lo stetoscopio sente se \`e presente la sorgente di rumore scelta. 
	Questo modello acustico \`e necessariamente un modello approssimato. 
	In una prima fase elementare di modellazione possiamo usare un semplice filtro attenuatore e supporre che il rumore sia di tipo additivo.
      \item
	Scegliere un file contente un respiro.
      \item
	Fare un mix dei file.
    \end{enumerate}
  \item
    Suoni respiratori senza rumore. 
    Lo scopo di un caso d'uso con questo input \`e la valutazione del funzionamento del software in uno scenario ideale.
%   \item
%     Un file audio contenente solo rumore. 
%     Questo caso di test serve per capire se il software pu\`o rilevare la presenza di respiro in suoni che non contengono alcun respiro. In uno scenario di uso reale corretto questo caso non si verifica ma \`e comunque interessante.
%   \item
%     Un file con una frequenza di campionamento molto elevata. 
%     Questo caso di test rientra nella categoria stress test. 
%     Ci aspettiamo che il sistema si comporti bene se ha un input file con una frequenza di campionamento molto elevata grazie al filtro di sottocampionamento.
  \item
    Un file contenente suoni respiratori sovrapposti a forti suoni cardiaci. 
  \item
    Un file contenente suoni respiratori e una apnea pi\`u lunga della soglia massima.
\end{itemize}


In generale possiamo creare un caso di test attraverso la concatenazione di segmenti di file audio ognuno con una possibile configurazione di propriet\`a secondo quanto illustrato nella tabella.
Non abbiamo a disposizione alcuno stetoscopio per\`o usiamo alcune registrazione reperite online e partiamo da queste per creare alcuni casi di test.
