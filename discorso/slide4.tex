\section{Slide 4 Respirazione e apnea}

La \emph{respirazione} \`e il processo che consente lo scambio di aria tra i polmoni e l'ambiente circostante. 
La respirazione permette di acquisire ossigeno nel sangue e di eliminare i residui gassosi del metabolismo come l'anidride carbonica. 
La respirazione \`e ciclica e un ciclo di respirazione ha atto in tre fasi consecutive:
\begin{description}
  \item[Inspirazione.]
    Durante questa fase i muscoli respiratori (diaframma, muscoli intercostali, ...) si contraggono e fanno aumentare di volume i polmoni.
    La pressione intrapleurica diminuisce e ne consegue una aspirazione dell'aria nei polmoni.
  \item[Espirazione.]
    Durante questa fase i muscoli respiratori si rilassano, i polmoni rilasciano l'energia elastica e tornano nella posizione iniziale.
    L'aria viene espulsa dai polmoni. 
    \item[Pausa.]
    Tra una espirazione e l'inspirazione successiva ci pu\`o essere una pausa di durata variabile. 
%     In realt\`a ci pu\`o essere una pausa anche tra l'inspirazione e l'espirazione ma durante il sonno, la pausa successiva all'espirazione \`e relativamente pi\`u lunga.
\end{description}

Una \emph{apnea} \`e una pausa di durata anormale nella respirazione che supera i dieci secondi  e pu\`o durare anche alcuni minuti. 
