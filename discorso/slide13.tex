\section{Slide 13 Riconoscimento delle apnee}
Dopo la fase di filtraggio passiamo alla fase di risconoscimento.
L'energia del segnale \`e definita semplicemente come la somma dei quadrati dei campioni.
Ci chiediamo se l'energia in una finestra rettangolare di $10ms$ sia maggiore dell'energia media nella finestra rettangolare di $4s$ che la contiene.
Se \`e maggiore allora il suono nella finestra di segnale viene classificata come inspiratorio o espiratorio, altrimenti il suono viene classificato come pausa respiratoria.
In seguito troviamo la fase di clustering che serve per correggere eventuali errori, ad esempio un picco nel suono che non era causato dal respiro e che non \`e stato eliminato dalla fase di filtraggio oppure serve per unire la fase di espirazione con quella di inspirazione in quanto potrebbe esserci una pausa molto breve tra di esse.