\section{Slide 20 Sviluppi futuri}

    L'applicazione pu\`o essere modificata per:
    \begin{itemize}
      \item
	Migliorare il riconoscimento delle fasi respiratorie.
      \item
	Aggiungere la possibilit\`a di classificare i suoni respiratori (normali, anormali, soffi, crackles, ...).
      \item
	Estendere l'applicazione con la funzione di riconoscere gli schemi respiratori (normale, Cheyne Stokes, agonico, Kussmaul, ...).
    \end{itemize}


    Si possono sviluppare alcuni pezzi mancanti dell'applicazione ad esempio:
    \begin{itemize}
      \item 
	Implementare una interfaccia con uno stetoscopio elettronico e fare dei test su soggetti affetti da sindrome di apnea del sonno.
      \item
	Implementare un algoritmo di stima del flusso respiratorio.
    \end{itemize}
  

    Un altro sviluppo futuro consiste nello studiale la portabilit\`a dell'applicazione su un dispositivo mobile, sia nel caso in cui si usa il microfono in dotazione del dispositivo che nel caso in cui il dispositivo riceva i dati da uno stetoscopio elettronico. 
    Per questo scopo una tecnologia da valutare \`e J2ME.


    Ci sono vari modi di procedere utili alla ricerca nell'ambito dell'analisi dei suoni respiratori, ad esempio:
    \begin{itemize}
      \item
	Creare un database di registrazioni di suoni respiratori.
      \item	
	Creare un database di casi di test completo. 
      \item	
	Creare un modello acustico approssimato del torace.
      \item
	Implementare dei meccanismi di tolleranza al rumore esterno. 
	Ad esempio usare due microfoni: uno che registra il rumore ambientale e uno che registra i suoni respiratori e usare il modello acustico per estrarre il rumore ambientale dai suoni respiratori.
      \item
	Fare una analisi approfondita dello stato dell'arte della separazione dei suoni cardiovascolari dai suoni respiratori, partendo ad esempio dagli articol
    \end{itemize}
