\section{Dimensione frattale}
\label{frattale}
In geometria frattale la dimensione frattale \`e una quantit\`a statistica che da una indicazione di quanto completo appare un frattale per riempire lo spazio. 
La dimensione frattale inoltre \`e una misura della complessit\`a di un insieme di dati. 
Viene usata per analizzare segnali in un vasto numero di ricerche scientifiche. 
Una propriet\`a della dimensione frattale \`e che \`e indipendente dal contenuto di energia nel segnale ma \`e dipendente dalle frequenze. 
Ci sono molti modi di definire matematicamente la dimensione frattale ad esempio quello illustrato in \cite{RSDUVFD}:
\[
  D_{\sigma} = \displaystyle\lim_{\delta t \rightarrow 0} \frac{\log(Var(\delta S, \delta t))}{2 \cdot \log(\delta t)}
\]
In questa formula: $\delta t$ \`e il valore assoluto di una variazione temporale $|t_{2}-t_{1}|$ mentre $\delta S$ \`e la variazione del segnale nel tempo $S(t_{2})-S(t_{1})$.



 