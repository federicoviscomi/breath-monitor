\chapter{Introduzione}

\section{Scopo}

Lo scopo di questa tesi \`e di implementare un prototipo di un meccanismo di monitoraggio della respirazione usando uno stetoscopio elettronico. 
Il punto di arrivo \`e un software che prende in input il segnale di uno stetoscopio elettronico e capisce se il soggetto sta respirando o no.
Il viaggio per arrivare a questa meta passa anche da una esplorazione dello stato dell'arte, una tappa di per se importante.



\section{Motivazioni}

La sindrome da apnea del sonno \`e una patologia molto diffusa. 
Il sonno di un soggetto affetto da tale patologia \`e disturbato da apnee e da episodi di respirazione insufficiente. 
Forme medie e gravi di sindrome da apnea del sonno sono un fattore di rischio per, e una concausa di: pressione alta, malattie cardiache, diabete, depressione. 
Inoltre tale patologia contribuisce a creare una senso perenne di sonnolenza e spossatezza. 
Si stima che pi\`u della met\`a dei soggetti affetti da tale patologia non ne siano al corrente\cite{intrrr}. 
Questo a causa dei meccanismi di diagnosi pi\`u diffusi che sono costosi o scomodi e richiedono al paziente di trascorrere la notte in un centro specializzato.
Inoltre i centri specializzati sono pochi e possono essere molto costosi. 
Sono in fase di sviluppo ma non hanno per il momento una diffusione capillare altri strumenti di diagnosi pi\`u pratici. 
Alcuni studi hanno analizzato i dati di alcuni soggetti che sono morti a causa di eventi cardiovascolari acuti e che erano 
affetti da forme medie o gravi di sindrome da apnea del sonno. 
La conclusione \`e stata che la maggior parte di tali soggetti \`e morta durante il sonno. 
In questa tesi sviluppiamo un prototipo di uno strumento non invasivo di monitoraggio del respiro e di diagnosi della sindrome da apnea del sonno: attraverso uno stetoscopio elettronico, il sistema deve vigilare su un soggetto e registrare la frequenza e la durata delle apnee e, cosa pi\`u importante, deve svegliare il soggetto nel caso in cui l'apnea duri troppo. 




\section{Contenuto della tesi}

La tesi \`e divisa in tre parti: 
\begin{enumerate}
  \item
    La prima parte contiene:
    \begin{itemize}
      \item
	I requisiti necessari alla comprensione del problema.
      \item
	I requisiti necessari alla comprensione delle tecniche di soluzione del problema. Le tecniche di soluzione sono: 
	\begin{itemize}
	  \item
	    Quelle di cui si parla nello stato dell'arte.
	  \item
	    Quelle usate nel sistema software implementato. 
	  \item
	    Quelle di cui \`e stata valutata la fattibilit\`a con esito positivo. 
	\end{itemize}
    \end{itemize}
    Questa parte parla di tutti gli argomenti necessari, con un livello di sintesi proporzionale all'importanza dell'argomento.
  \item
    La seconda parte contiene: 
    \begin{itemize}
      \item
	Un riassunto dello stato dell'arte e di come si pu\`o adattare il materiale presente nello stato dell'arte per risolvere il problema.
      \item
	Una analisi e una discussione delle metodologie in atto e in potenza per la risoluzione del problema.
    \end{itemize}
  \item
    La terza parte contiene la descrizione di una implementazione di un prototipo di un sistema che risolve il problema.
\end{enumerate}